\begin{frame}
	\frametitle{Material}
	
	Material para armar estas slides:
	\begin{itemize}
		\item Cyrill Stachniss - Robot Motion Planning using A*: \url{https://youtu.be/HR1TNa8Lp7w}
	\end{itemize}
	
\end{frame}

\begin{frame}
	\frametitle{Planeamiento de movimiento}

	\begin{block}{Motion Planning}
		Planificando movimientos para que los robots se muevan del punto 𝐴 al punto 𝐵 (como para robots móviles, etc.) o posen 𝐴 para posar 𝐵 (como para manipuladores, etc.). Para hacerlo, es necesario tener en cuenta una serie de restricciones: prevención de colisiones, límites de juntas, límites de velocidad/aceleración, límites de tirones, equilibrio dinámico, límites de par y muchos más. En este sentido, no solo se considera el robot sino también su entorno (p. ej., para evitar colisiones, cómo mantener el equilibrio). Teniendo esto en cuenta, la planificación de movimiento es una especie de generación de trayectoria con muchas restricciones.
	\end{block}

\end{frame}