\begin{frame}
	\frametitle{Material}
	
	Material para armar estas slides:
	\begin{itemize}
		\item Cyrill Stachniss - Robot Motion Planning using A*: \url{https://youtu.be/HR1TNa8Lp7w}
		\item Wolfram Burgard - Path and Motion Planning \url{http://ais.informatik.uni-freiburg.de/teaching/ss18/robotics/slides/19-pathplanning-long.pdf}
		\url{http://ais.informatik.uni-freiburg.de/teaching/ss16/robotics/recordings/19-path-and-motion-planning-part1.mp4}
		\url{http://ais.informatik.uni-freiburg.de/teaching/ss16/robotics/recordings/19-path-and-motion-planning-part2.mp4}
		\url{http://ais.informatik.uni-freiburg.de/teaching/ss16/robotics/recordings/19-path-and-motion-planning-part3.mp4}
	\end{itemize}
	
\end{frame}

\begin{frame}
	\frametitle{Motion Planning}
	\note{Información extraída de http://ais.informatik.uni-freiburg.de/teaching/ss18/robotics/slides/19-pathplanning-long.pdf}
	
	Latombe (1991): ``... eminently necessary since, by definition, a robot accomplishes tasks by moving in the real world.''
	
	Objetivos:
	\begin{itemize}
		\item Trayectorias sin colisiones.
		\item El robot debe llegar a la ubicación de destino lo más rápido posible.
	\end{itemize}
\end{frame}

\begin{frame}
	\frametitle{Desafíos}
	\note{Información extraída de http://ais.informatik.uni-freiburg.de/teaching/ss18/robotics/slides/19-pathplanning-long.pdf}
	
	\begin{itemize}
		\item Calcular el camino optimo considerando potenciales incertidumbres en las acciones
		\item Generar rápidamente acciones en el caso de objetos imprevistos
	\end{itemize}
	
\end{frame}


\begin{frame}
	\frametitle{Arquitectura clásica}
	\note{Información extraída de http://ais.informatik.uni-freiburg.de/teaching/ss18/robotics/slides/19-pathplanning-long.pdf}
	
	\begin{figure}[!h]
		\includegraphics[width=0.3\textwidth]{images/path_planning_architecture.pdf}
	\end{figure}
	
\end{frame}


\begin{frame}
	\frametitle{Motion Planning Formulation}
	\note{Información extraída de http://ais.informatik.uni-freiburg.de/teaching/ss18/robotics/slides/19-pathplanning-long.pdf}
	
\end{frame}


\begin{frame}
	\frametitle{Espacio de configuraciones}
	
	
	Espacio de configuraciones es el estado total del robot y el entorno, incluidas las articulaciones y/o movimiento de ruedas)
	
\end{frame}

\begin{frame}
	\frametitle{Motion Planning Formulation}
	
\end{frame}


\begin{frame}
	\frametitle{Planeamiento de movimiento}
	
	\begin{block}{Motion Planning}
		Problema de encontrar rutas sin colisiones a través del espacio de configuraciones. Los obstáculos son convertidos (implícitamente con un verificador de colisiones booleano) al espacio de configuraciones. Se utilizan varios algoritmos geométricos o aleatorios/probabilísticos para buscar a través del espacio de configuraciones un camino desde el principio hasta la meta que no choque con ningún obstáculo.
	\end{block}

	\begin{block}{Trajectory Planning}
		Una vez que se tiene un plan de movimiento, este es simplemente un camino geométrico a través del espacio, {\bf no está asociado a un tiempo}. Hay infinitas trayectorias por  cada camino, ya que se puede mover a través del camino con patrones de velocidad variables ilimitados. Durante el planeamiento de movimiento, solo se desea que el movimiento cumpla con algunas restricciones físicas, como qué tan rápido se puede mover el robot y que sea lo más suave posible (por razones de control). Sin embargo, si hay obstáculos dinámicos en el entorno, se tiene que introducir el tiempo en la planificación (el tiempo es una dimensión monótona, es decir, solo puede avanzar, no retroceder). Entonces, el camino debe especificar el momento correcto; de lo contrario, el camino no será válido porque el robot podría colisionar con algunos de los objetos dinámicos.
	\end{block}

	\begin{block}{Path Planning}
	Un camino NO es igual a una trayectoria. Una trayectoria es un camino con información adicional de cómo se atravesar dicho camino con respecto al tiempo, es decir tiene información de velocidad.
	\end{block}
\end{frame}


\begin{frame}
	\frametitle{Rapidly Exploring Random Trees (RTTs)}
	
\end{frame}

\begin{frame}
	\frametitle{Road Map Planning}
	
\end{frame}


\begin{frame}
	\frametitle{Generalized Voronoi Diagram}
	
\end{frame}

\begin{frame}
	\frametitle{Voronoi Diagram}
	
\end{frame}

\begin{frame}
	\frametitle{Randomized Road Maps}
	
\end{frame}

\begin{frame}
	\frametitle{From Road Maps to Paths}
	
\end{frame}

\begin{frame}
	\frametitle{Randomized Road Maps}
	
\end{frame}

\begin{frame}
	\frametitle{Randomized Road Maps}
	
\end{frame}

\begin{frame}
	\frametitle{Randomized Road Maps}
	
\end{frame}

\begin{frame}
	\frametitle{Randomized Road Maps}
	
\end{frame}

\begin{frame}
	\frametitle{Markov Decision Process}
	
\end{frame}

