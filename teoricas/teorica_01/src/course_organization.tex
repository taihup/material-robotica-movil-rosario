\begin{frame}
	\frametitle{Programa de la materia (Parte I)}
	\footnotesize
	\begin{enumerate}
		\item {\bf Introducción:} Historia de la Robótica Móvil. Tipos de Robots. Campos de aplicación de la Robótica Móvil. Desafíos de la Robótica Móvil.
		
		\item {\bf Percepción:} tipos de sensores. Sensores interoceptivos y exteroceptivos. Modelo de sensores. Ventajas y desventajas de cada tipo de sensor. Caracterización del ruido.
		
		\item {\bf Cinemática:} Sistemas de locomoción. Modelo Diferencial. Modelo de Ackerman. Holonómico/No-Holonómico.
		
		\item {\bf ROS y Gazebo:} Introducción a ROS 2. Herramientas de visualización y depuración. Recolección de Datos. Simulador Gazebo.
		
		\item {\bf Visión en Robótica:} Geometría proyectiva. Extracción de características visuales. Calibración Visual: intrínseca y Extrínseca. 
		
	\end{enumerate}

\end{frame}

\begin{frame}
	\frametitle{Programa de la materia (Parte II)}
	\footnotesize
	\begin{enumerate}

		\item {\bf Localización:} Modelo probabilístico. Teoría de Bayes. Principio de independencia de Markov. Filtros Gausianos: Filtro Extendido de Kalman. Filtros no-paramétricos: Monte Carlo e Histograma.

		\item {\bf Mapeo:} Nube de puntos. Grilla de Ocupación, árbol cuaternario (Quadtree) y árbol octal (Octree), campos de distancia de signo truncado (TSDF).
		
		\item {\bf Localización y Mapeo Simultáneo (SLAM):} Grafo de Factores. Métodos de optimización: Método de descenso por gradiente, Método Gauss-Newton, Método Levenberg-Marquardt y Bundle Adjustment. Grupos de Lie. Álgebra de Lie. Pre-integración. Problema del Robot Secuestrado (Kidnapped Robot Problem). Relocalización. Detección y Cierre de Ciclos.
		
		\item {\bf Planeamiento de Caminos:} Algoritmo A*. Algoritmo de Dijkstra. Grafo de Visibilidad. Descomposición de celdas. Diagrama de Voronoi. Campos de potencial artificial, Probabilistic RoadMap, Rapidly Exploring Random Tree (RRT) y Rapidly-exploring Random Graph (RRG).
		
		\item {\bf Control:} Controlador proporcional-integral-derivativo (PID). Regulador Lineal Cuadrático (LQR). Control Predictivo por Modelo (MPC).
		
	\end{enumerate}

\end{frame}

\begin{frame}
	\frametitle{Métodología de evaluación}
	
	Correlativas:
	\begin{itemize}
		\item 1er cuatrimestre de 4to año aprobado.
	\end{itemize}

	
	Regularizar:
	\begin{itemize}
		\item Entregas + Trabajos Prácticos (en grupos de 2)
		\item 1 Parcial
	\end{itemize}

	Evaluación final:
	\begin{itemize}
		\item Trabajo Práctico + coloquio (en grupos de 2)
	\end{itemize}	
\end{frame}