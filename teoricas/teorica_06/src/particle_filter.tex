\begin{frame}
    \frametitle{Filtro de Partículas (Particle Filter)}
    \TODO{Vídeo de Cyrill Stachniss \url{https://youtu.be/MsYlueVDLI0}}
    
    \TODO{UTILIZAR LAS SLIDES DEL SEMINARIO}
    
    \begin{itemize}
        \item Con EKF estamos restringidos a distribuciones Gaussianas.
    \end{itemize}
    
\end{frame}


\begin{frame}
    \frametitle{Filtro de Partículas (Particle Filter)}
    \TODO{Vídeo de Cyrill Stachniss \url{https://youtu.be/MsYlueVDLI0}}
    
    \TODO{UTILIZAR LAS SLIDES DEL SEMINARIO}
    
    \begin{itemize}
        \item Cuando usamos EKF obtenemos una Distibuión Gaussiana que describe dónde se encuentra el robot.
        \item En Particle Filter utilizamos partículas o hipótesis que describen dónde podría estar el robot.
        \item En vez de tener una forma paramétrica como es EKF, que describimos la distribución de probabilidad con los parámetros media $\mu$ y covarianza $\covariance$. Partible Filter utiliza muestras no-paramétricas como hipótesis sobre dónde el robot podría estar.
    \end{itemize}
    
\end{frame}
