\begin{frame}
    \frametitle{Material}
    
    Vídeos para armar estas slides:
    \begin{itemize}
        \item Cyrill Stachniss Bayes: \url{https://youtu.be/0lKHFJpaZvE}
        \item Cyrill Stachniss KF y EKF: \url{https://youtu.be/E-6paM_Iwfc}
        \item Chebrolu EKF Localization: \url{https://youtu.be/PiCC-SxWlH8}
        \item Cyrill Stachniss Particle Filter: \url{https://youtu.be/MsYlueVDLI0}
        \item Slides de ECI 2012
        \item Seminario de curso: CS373 Udacity Programming a Robotic Car
    \end{itemize}
    
\end{frame}

\begin{frame}
    \frametitle{Temario para estas slides}
    
    \begin{itemize}
        \item Bayes filter
        \item kalman filter
        \item Particle filter
    \end{itemize}
    
    
\end{frame}

\begin{frame}
    \frametitle{Bayes Filter}
    
    \note{Información obtenida de Cyrill Stachniss Bayes: https://youtu.be/0lKHFJpaZvE}
   
    
\end{frame}

\begin{frame}
    \frametitle{Extended Kalman Filter}
    
    \note{Información obtenida de:
        Cyrill Stachniss KF y EKF: https://youtu.be/E-6paM_Iwfc
        Chebrolu EKF Localization: https://youtu.be/PiCC-SxWlH8}
\end{frame}


\begin{frame}
    \frametitle{índice}
    \tableofcontents%[pausesections]
\end{frame}

\section{Problema de localización}
\begin{frame}{Localización}
    \begin{block}{Localización}
        Es la habilidad que posee una máquina para localizarse en el espacio.
    \end{block}
\end{frame}


\begin{frame}{Ejemplo de localización}
    \begin{block}{Ejemplo}
        \begin{itemize}
            
            \item Dado un robot en un mundo de una dimensión, sin conocimiento de en donde se encuentra
            \item El robot se puede mover hacia delante o atrás
            \item Supongamos además que hay tres puertas (\alert{landmarks}), el robot puede detectar si se encuentra al lado de una puerta o no.
        \end{itemize}
    \end{block}
\end{frame}

\begin{frame}{Ejemplo: Posición inicial}
    Como en un principio el robot desconoce cual es su posición entonces es igualmente posible que se encuentre en cualquier punto del mundo (\alert{belief}). Esto lo podemos representar matemáticamente diciendo que la \alert{función de distribución de probabilidad} del robot es \alert{uniforme} sobre el mundo en que se encuentra.
    \begin{center}
        \includegraphics<1>[height=2cm]{./images/monte_carlo_uniform.png}
    \end{center}
    
\end{frame}

\begin{frame}{Ejemplo: Medición}
    
    Si el robot sensa que se encuentra al lado de una puerta entonces la creencia de su ubicación se se ve alterada de la siguiente manera:
    
    \begin{center}
        \includegraphics<1>[height=2cm]{./images/monte_carlo_sensing.png}
    \end{center}
    
    
    Esta nueva función representa otra distribución de probabilidades llamada \alert{Posterior belief}.
    
    La función Posterior belief es la mejor representación de la posición del robot actualmente. Cada ``loma'' representa la evaluación de su posición con respecto a una puerta.
    
\end{frame}

\begin{frame}{Ejemplo: Movimiento}
    
    Si el robot se mueve hacia la derecha la creencia es cambiada de acuerdo al movimiento.
    Así mismo como el movimiento del robot es inexacto, al trasladarse su incertidumbre crece, dicho de otra manera, las lomas son aplanadas. Este aplanamiento matemáticamente se lleva a cabo por medio de la operación de \alert{convolución} entre la función Posterior belief y la función que describe la distancia recorrida.
    
    \begin{center}
        \includegraphics<1>[height=2cm]{./images/monte_carlo_moving.png}
    \end{center}
    
    La operación de convolución mide la superposición mientras se desliza una funcion sobre otra.
    
\end{frame}

\begin{frame}{Ejemplo: Segunda medición}
    Supongamos que el robot luego de haberse movido sensa nuevamente que se encuentre al lado de una puerta entonces, como antes, la probabilidad se incrementara por un cierto factor la función de probabilidad donde haya una puerta.
    
    \begin{center}
        \includegraphics<1>[height=2cm]{./images/monte_carlo_sensing2.png}
    \end{center}
\end{frame}

\section{Medición}

\begin{frame}{Belief luego de sensar}
    \begin{block}{Ejemplo}
        \begin{itemize}
            
            \item Un mundo constituido por cinco celdas $x_{i}$ donde $i = 1, \dots ,5$
            \item Las celdas $x_{2}$ y $x_{3}$ son rojas, y el resto verdes.
            \item Inicialmente el robot desconoce su posición
            \item La probabilidad de que el robot sense correctamente  esta dada por la siguiente distribución de probabilidades:
            
            \begin{displaymath}
                P(sensa \ color | x_{i} = color) = 0.6
            \end{displaymath}
            \begin{displaymath}
                P(sensa \ \neg color | x_{i} = color) = 0.2
            \end{displaymath}
            
            Observar que no es una distribución de probabilidad correcta ya que la suma debe ser $1$.
            
        \end{itemize}
        
    \end{block}
    
    \begin{center}
        \includegraphics<1>[height=1.0cm]{./images/uniform_five_cells.png}
    \end{center}
    
\end{frame}

\begin{frame}{Belief luego de sensar}
    
    Si el robot \alert{sensa rojo}, ?`Cuál es su Posterior belief?
    
    \begin{center}
        \includegraphics<1>[height=3.5cm]{./images/inaccurate_sensing_quiz.png}
    \end{center}
\end{frame}

\begin{frame}{Belief luego de sensar}
    
    Si el robot \alert{sensa rojo}, ?`Cuál es su Posterior belief?
    
    \begin{center}
        \includegraphics<1>[height=3.5cm]{./images/inaccurate_sensing_solution.png}
    \end{center}
    \begin{footnotesize}
        \begin{displaymath}
            P(x_{i} = rojo | sensa \ rojo) = P(sensa \ rojo | x_{i} = rojo) P(x_{i}) = 0.2 \times 0.6 = 0.12
        \end{displaymath}
        \begin{displaymath}
            P(x_{i} = verde | sensa \ rojo) = P(sensa \ rojo | x_{i} = verde) P(x_{i}) = 0.2 \times 0.2 = 0.04
        \end{displaymath}
    \end{footnotesize}
    
\end{frame}

\begin{frame}{Belief luego de sensar}
    \begin{displaymath}
        P(x_{i} = rojo | sensa \ rojo) = 0.12
    \end{displaymath}
    \begin{displaymath}
        P(x_{i} = verde | sensa \ rojo) = 0.04
    \end{displaymath}
    
    Observar que estamos ante una distribución de probabilidades formalmente incorrecta dado que la suma: 
    \begin{displaymath}
        \sum_{i=1}^{5} P(x_{i}) = 0.04 + 0.12 + 0.12 + 0.04 + 0.04 = 0.36
    \end{displaymath}
    
    
    Si normalizamos la distribución, queda:
    
    \begin{displaymath}
        P(x_{i} = rojo| sensa \ rojo) = \dfrac{0.12}{0.36} = \dfrac{1}{3}
    \end{displaymath}
    \begin{displaymath}
        P(x_{i} = verde| sensa \ rojo) = \dfrac{0.04}{0.36} = \dfrac{1}{9}
    \end{displaymath}
    
    En general, $P(x_{i}|z)$ es la distribución Posterior belief del lugar $x_{i}$ dada la medición $z$.
\end{frame}


\begin{frame}{Regla de Bayes}
    
    Notar que cuando el robot sensa no hace otra cosa que aplicar la Regla Bayes:
    
    \begin{block}{Regla de Bayes}
        \begin{displaymath}
            P(x_{i} | z) = \dfrac{P(z | x_{i})P(x_{i})} {P(z)} 
        \end{displaymath}
        $P(x_{i} | z)$ : probabilidad a Posteriori (Posterior Belief) \\
        $P(z | x_{i})$ : probabilidad de Medición \\
        $P(x_{i})$ : probabilidad a Priori \\
        $P(z)$ : término de Normalización
    \end{block}
    
\end{frame}

\section{Motricidad}
\begin{frame}{Belief luego del movimiento}
    \begin{block}{Ejemplo (continuación)}
        \begin{itemize}
            \item Un mundo \alert{cíclico} constituido por cinco celdas $x_{i}$ donde $i = 1, \dots ,5$
            \item La distribución de probabilidad a priori esta determinada por:
            \begin{displaymath}
                P(x_{1}) = P(x_{4}) = P(x_{5}) = \dfrac{1}{9}
            \end{displaymath}
            \begin{displaymath}
                P(x_{2}) = P(x_{3}) = \dfrac{1}{3}	
            \end{displaymath}
        \end{itemize}
    \end{block}
    
\end{frame}

\begin{frame}{Belief luego del movimiento}
    Si el robot tiene una \alert{motricidad exacta} y desea moverse \alert{una} celda a la derecha, ?`Cuál es su Posterior belief?
    
    \begin{center}
        \includegraphics<1>[height=3.5cm]{./images/exact_motion_quiz.png}
    \end{center}
    
\end{frame}

\begin{frame}{Belief luego del movimiento}
    Si el robot tiene una \alert{motricidad exacta} y desea moverse \alert{una} celda a la derecha, ?`Cuál es su Posterior belief?
    
    \begin{center}
        \includegraphics<1>[height=3.5cm]{./images/exact_motion_solution.png}
    \end{center}
    
\end{frame}

\begin{frame}{Belief luego del movimiento}
    Suponiendo ahora que el robot desea moverse \alert{dos} celdas a la derecha y tiene una \alert{motricidad inexacta} con la siguiente distribución de probabilidad:
    \begin{columns}[t]
        \begin{column}{5cm}
            \begin{displaymath}
                P(x_{i+2}| x_{i}) = 0.8
            \end{displaymath}
            \begin{displaymath}
                P(x_{i+1}| x_{i}) = 0.1
            \end{displaymath}
            \begin{displaymath}
                P(x_{i+3}| x_{i}) = 0.1
            \end{displaymath}
        \end{column}
        \begin{column}{5cm}
            \begin{center}
                \includegraphics<1>[height=1.8cm]{./images/inexact_motion.png}
            \end{center}
        \end{column}
    \end{columns}
\end{frame}

\begin{frame}{Belief luego del movimiento}
    
    Si el robot conoce exactamente cuál es su posición inicial, ?`Cuál es su Posterior belief?
    
    \begin{columns}[t]
        \begin{column}{5cm}
            \begin{center}
                \includegraphics<1>[height=2.0cm]			{./images/inexact_motion_initial_pose_quiz.png}
            \end{center}
        \end{column}
        \begin{column}{5cm}
            \begin{center}
                \includegraphics<1>[height=1.8cm]{./images/inexact_motion.png}
            \end{center}
        \end{column}
    \end{columns}
\end{frame}

\begin{frame}{Belief luego del movimiento}
    
    Si el robot conoce exactamente cuál es su posición inicial, ?`Cuál es su Posterior belief?
    
    \begin{columns}[t]
        \begin{column}{5cm}
            \begin{center}
                \includegraphics<1>[height=2cm]			{./images/inexact_motion_initial_pose_solution.png}
            \end{center}
        \end{column}
        \begin{column}{5cm}
            \begin{center}
                \includegraphics<1>[height=1.8cm]{./images/inexact_motion.png}
            \end{center}
        \end{column}
    \end{columns}
\end{frame}

\begin{frame}{Belief luego del movimiento}
    Si el robot tiene como distribución inicial que se encuentra en las celdas $x_{2}$ y $x_{4}$ con igual probabilidad, formalmente,
    \begin{displaymath}
        P(x_{2}) = P(x_{4}) = 0.5
    \end{displaymath}
    ?`Cuál es su Posterior belief?
    
    \begin{columns}[t]
        \begin{column}{5cm}
            \begin{center}
                \includegraphics<1>[height=2cm]{./images/inexact_motion_quiz.png}
            \end{center}
        \end{column}
        \begin{column}{5cm}
            \begin{center}
                \includegraphics<1>[height=1.8cm]{./images/inexact_motion.png}
            \end{center}
        \end{column}
    \end{columns}
\end{frame}

\begin{frame}{Belief luego del movimiento}
    
    Si el robot tiene como distribución inicial que se encuentra en las celdas $x_{2}$ y $x_{4}$ con igual probabilidad, formalmente,
    \begin{displaymath}
        P(x_{2}) = P(x_{4}) = 0.5
    \end{displaymath}
    ?`Cuál es su Posterior belief?
    
    \begin{columns}[t]
        \begin{column}{5cm}
            \begin{center}
                \includegraphics<1>[height=2cm]{./images/inexact_motion_solution.png}
            \end{center}
        \end{column}
        \begin{column}{5cm}
            \begin{center}
                \includegraphics<1>[height=1.8cm]{./images/inexact_motion.png}
            \end{center}
        \end{column}
    \end{columns}
    \begin{small}
        $P(x_{1}) = P(x_{4}) P(x_{1}|x_{4}) = 0.5 \times 0.8 = 0.4$ \\
        $P(x_{2}) = P(x_{4}) P(x_{2}|x_{4}) = 0.5 \times 0.1 = 0.05$ \\
        $P(x_{3}) = P(x_{2}) P(x_{3}|x_{2}) = 0.5 \times 0.1 = 0.05$ \\	
        $P(x_{4}) = P(x_{2}) P(x_{4}|x_{2}) = 0.5 \times 0.8 = 0.4$ \\	
        $P(x_{5}) = P(x_{2}) {\color{red} P(x_{5}|x_{2})} + P(x_{4}) {\color{green} P(x_{5}|x_{4})} = 0.5 \times {\color{red} 0.1} + 0.5 \times {\color{green} 0.1} = 0.1$	
    \end{small}
\end{frame}

\section{Sensar y Mover}
\begin{frame}{Ciclo de sensar y mover}
    Localización no es más que la iteración de sensar y mover.
    \begin{center}
        \includegraphics<1>[height=2.5cm]{./images/sens_and_move.pdf}
    \end{center}
    
    \begin{block}{Entropía}
        Medida de información que tiene la distribución
        \begin{displaymath}
            - \sum P(x_{i}) \log P(x_{i})
        \end{displaymath}
        
        En otras palabras, la entropía expresa la información que un robot recibe luego de ejecutar una acción específica.
    \end{block}
    
\end{frame}




\begin{frame}{Definición formal de localización}
    
    \begin{block}{Medición}
        \begin{displaymath}
            \Bar{P}(x_{i}|z) \leftarrow P(z|x_{i}) P(x_{i})
        \end{displaymath}
        \begin{displaymath}
            \alpha \leftarrow \sum \Bar{P}(x_{i}|z)
        \end{displaymath}
        \begin{displaymath}
            P(x_{i}|z) \leftarrow \frac{1}{\alpha} \Bar{P}(x_{i}|z)
        \end{displaymath}
        
    \end{block}
    
\end{frame}

\begin{frame}{Definición formal de localización}
    
    Sea $P(x_{i}^{t})$ la probabilidad de estar en el punto $x_{i}$ luego del movimiento del robot
    
    \begin{block}{Motricidad}
        
        \begin{displaymath}
            P(x_{i}^{t}) = \sum_{j} P(x_{j}^{t-1}) P(x_{i}|x_{j})
        \end{displaymath}
        
    \end{block}
    
    La probabilidad de estar en $x_{i}$ se calcula a través de todos los lugares de los que podríamos haber venido
    
    Observar que la expresión anterior no es otra cosa que el Teorema de Probabilidad total.
    
    \begin{block}{Teorema de Probabilidad Total}
        
        \begin{displaymath}
            P(A) = \sum_{B}^{}P(A|B) P(B)
        \end{displaymath}
        
    \end{block}
    
\end{frame}




