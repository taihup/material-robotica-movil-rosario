% Estas slides tienen que abrirse con el programa pdfpc que soporta videos embebidos
% el comando es: pdfpc -g slides.pdf
% para los videos se requiere ubuntu-restricted-extras
% para la bibliografía se requiere biber y configurar texstudio


%\documentclass[compress,handout]{beamer}
\documentclass[compress]{beamer}


% Theme customization
\setbeamertemplate{itemize item}[rectangle] % configure itemize
\setbeamertemplate{itemize subitem}[circle] % configure itemize
\setbeamertemplate{itemize subsubitem}[triangle] % configure itemize
\setbeamertemplate{navigation symbols}{} % remover simbolos de navegacion de las slides
\usefonttheme[onlymath]{serif} % simbolos matematicos en serif (Como es en latex original)

\setbeamertemplate{blocks}[rounded] % blocks corners rounded
\setbeamercolor{block body}{bg=blue!12,fg=black} % color of blocks

% Latex packages
\usepackage{pdfpc-commands} % pdfpc movie commands. Requires to be downloaded manually from https://github.com/pdfpc/pdfpc/issues/328#issuecomment-437397536
\usepackage[utf8]{inputenc}
\usepackage[T1]{fontenc} % para copiar acentos en español del pdf y permite acentos en las notas
\usepackage[spanish]{babel}
\usepackage[binary-units=true,per-mode = symbol]{siunitx} % para manejar las unidades
\usepackage{multirow}
\usepackage{graphicx}
\usepackage{xcolor}
\usepackage{amsmath} % bmatrix
\usepackage[caption=false]{subfig} % caption = false elimina la palabra "Figura" del caption
\setbeamertemplate{caption}{\raggedright\insertcaption\par} % elimina la palabra "Figura" del caption
\usepackage{import} % para el comando import (se usa para pdf_tex)
\captionsetup[subfigure]{labelformat=empty} % remover el indice del caption de la subfigura
\usepackage{booktabs} % \toprule \midrule \bottomrule
\usepackage[overridenote]{pdfpc} % requires to download manually pdfpc.sty package from https://www.ctan.org/pkg/pdfpc
\usepackage[backend=biber]{biblatex} % set biber to format references. Must configure Biber in Texstudio
\usepackage{csquotes} % to remove warning triggered by biblatex and babel

% Color commands for annotations
\newcommand\TODO[1]{\textbf{\textcolor{red}{#1}}} %  TODO notes

% add bibliography resource
\renewcommand*{\bibfont}{\footnotesize} % change bibliograhy size
\bibliography{./src/bibliography.bib}

% Graphic paths
\graphicspath{{./images/}}

% add math preamble
%\usepackage{amsmath}
\usepackage{amssymb}
\usepackage{amsopn}
\usepackage{mathtools}

% math
\renewcommand{\vec}[1]{\boldsymbol{\mathbf{#1}}}
\newcommand{\norm}[1]{\lVert#1\rVert}

% Declare arg max and arg min functionss
\DeclareMathOperator*{\argmax}{arg\,max}
\DeclareMathOperator*{\argmin}{arg\,min}

% Homogeneous decoration function
\newcommand{\homo}[1]{\dot{#1}}


% Declare projection as math function
\DeclareMathOperator{\proj}{proj}
\newcommand{\fromCoord}[2]{{#1}_\mathrm{#2}}
\newcommand{\toCoord}[2]{\prescript{\mathrm{#2}}{}{#1}}
\newcommand{\worldCoordSystem}{\mathrm{w}}
\newcommand{\bodyCoordSystem}{\mathrm{B}}
\newcommand{\cameraCoordSystem}{\mathrm{c}}
\newcommand{\point}{\vec{p}}
\newcommand{\worldPoint}{\toCoord{\point}{\worldCoordSystem}}
\newcommand{\imagePoint}{\vec{u}}
\newcommand{\cameraPoint}{\toCoord{\point}{\cameraCoordSystem}}
\newcommand{\homoWorldPoint}{\toCoord{\homo{\point}}{\worldCoordSystem}}
\newcommand{\homoImagePoint}{\homo{\imagePoint}}
\newcommand{\homoCameraPoint}{\toCoord{\homo{\point}}{\cameraCoordSystem}}
\newcommand{\measurement}{\vec{z}}
\newcommand{\prediction}{\hat{\vec{z}}}
\newcommand{\seMatrix}{\vec{\xi}}
\newcommand{\transform}[2]{\toCoord{\fromCoord{\seMatrix}{#2}}{#1}}
\newcommand{\pointCoord}[1]{\toCoord{\point}{#1}}
\newcommand{\rotation}{\vec{R}}
\newcommand{\rotationCoord}[2]{\toCoord{\fromCoord{\rotation}{#2}}{#1}}
\newcommand{\translation}{\vec{t}}
\newcommand{\translationCoord}[2]{\toCoord{\fromCoord{\translation}{#2}}{#1}}
\newcommand{\intrinsicMatrix}{\vec{K}}
\newcommand{\principalPoint}{\vec{c}}
\newcommand{\reprojectionError}{u}
\newcommand{\projectionMatrix}{\vec{P}}
\newcommand{\cameraCenter}{\vec{o}}
\newcommand{\essentialMatrix}{\vec{E}}
\newcommand{\inverse}[1]{{#1}^{-1}}

% Motion model
\newcommand{\position}{\vec{p}}
\newcommand{\orientationQuaternion}{\vec{q}}
\newcommand{\predictedPosition}{\hat{\vec{p}}}
\newcommand{\predictedOrientationQuaternion}{\hat{\vec{q}}}
\newcommand{\linearVelocity}{\vec{v}}
\newcommand{\angularVelocity}{\vec{\omega}}

\DeclareMathOperator{\slerpOp}{slerp}
\newcommand{\slerp}[1]{\slerpOp{\left( #1 \right)}}

% Map structure
\newcommand{\map}{M}
\newcommand{\keyframesSet}{K}
\newcommand{\mapPointsSet}{P}
\newcommand{\observedMapPoints}{O}
\newcommand{\covisibilityKeyframes}{CK}
\newcommand{\localMap}{local\_map}



% Bundle Adjutment
\newcommand{\update}{\vec{\delta}}
\newcommand{\incremental}{\hat{\update}}


% Loop Closure names

% scaled operators and letters to fancy view
\newcommand{\sminus}{\scalebox{0.5}[1.0]{$-$}}
\newcommand{\splus}{\scalebox{0.6}[0.6]{$+$}}
\newcommand{\curr}{c}
\newcommand{\sind}[1]{\scalebox{0.6}[0.6]{$#1$}}
\newcommand{\ind}[1]{\scalebox{0.7}[0.7]{$#1$}}

\newcommand{\keyframe}{\vec{K}}
\newcommand{\bowVector}{\vec{v}}
\newcommand{\lcError}{\vec{\Omega}}
\newcommand{\relativeTransformation}{\seMatrix}
\DeclareMathOperator{\interpolate}{interpolate}

\newcommand{\relativeMotion}{\vec{\delta}}
\newcommand{\groundTruth}[1]{{#1}^{*}}



% definición del operador rot()
\DeclareMathOperator{\rotationOp}{rot}
\newcommand{\getRotation}[1]{\rotationOp{\left( #1 \right)}}

\DeclareMathOperator{\translationOp}{trans}
\newcommand{\getTranslation}[1]{\translationOp{\left( #1 \right)}}









\title{Locomoción}
\author{}
\institute{Universidad Nacional de Rosario}
%\date{\scriptsize{Julio 1, 2021}}
\date{}

\begin{document}

% add title page
\frame{\titlepage}

\section{Tipos de locomoción}

\begin{frame}
    \frametitle{Locomoción}
    El humano copia a la naturaleza para construir tipos de locomoción.
    Mostrar ejemplos del libro Figura 2.1. animales y robots reales.
    
    \begin{itemize}
        \item Caminar
        \item Correr
        \item
    \end{itemize}
    
    La naturaleza tuvo que adaptar a los animales para terrenos irregulares!
    Los insectos por ejemplo tienen que sortear cambios de altura que superan en ordenes de magnistud su tamaño.
    
    Utilizamos ruedas por que son muy eficintes en terrenos planos y duros.
    
\end{frame}


\begin{frame}
    \frametitle{Locomoción}
    
    Manipulación: un brazo robótica está fijo y mueve objetos en su espacio de trabajo aplicando fuerza sobre ellos.
    
    Locomoción: el entorno está fijo y el robot se mueve impartiendo fuerza sobre el entorno.
    \begin{itemize}
        \item Estabilidad
        \begin{itemize}
            \item Geometría y Número de puntos de contacto
            \item Centro de gravedad
            \item Estabilidad estática y dinámica
            \item Inclinación del terreno
        \end{itemize}
        \item Característica de contacto
        \begin{itemize}
            \item Punto de contacto/ distancia y forma de camino
            \item Ángulo de contacto
            \item Fricción
        \end{itemize}
        \item Tipo de entorno
        \begin{itemize}
            \item Estructura
            \item Medio (agua, aire, terreno suave o firme)
        \end{itemize}
    \end{itemize}
\end{frame}

\begin{frame}
    \frametitle{Locomoción}
    Ventajas y desventajas de robots con patas:
    
    Ventajas: permite ir por terrenos irregulares, reducen el impacto ambiental dado que tienen menos puntos de contacto.
    Desventaja: complegidad mecánica, mayor consumo (una pata con muchos joints puede tener que bancar todo el paso del robot), para que tenga mucha maniobrabilidad cada las patas deben tener muchas juntas.
    
\end{frame}


\begin{frame}
    \frametitle{Gaits}
    
\end{frame}


\begin{frame}
    \frametitle{Tipos de Ruedas}

\end{frame}

\begin{frame}
    \frametitle{Drones?}
    
\end{frame}

\section{Cinemática de un robot móvil}
\begin{frame}
    \frametitle{Representación d Posición del Robot 2D}

\end{frame}


\begin{frame}
    \frametitle{Right-hand rule}

\end{frame}

\begin{frame}
    \frametitle{Sistema de referencia Local y Global}
    Poner imagen 3.1 del libro
\end{frame}

\end{document}
