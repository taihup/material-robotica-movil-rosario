% Estas slides tienen que abrirse con el programa pdfpc que soporta videos embebidos
% el comando es: pdfpc -g slides.pdf
% para los videos se requiere ubuntu-restricted-extras


%\documentclass[compress,handout]{beamer}
\documentclass[compress]{beamer}
% compress pone la seccion abajo de todo
\mode<presentation>


% Theme customization
\usetheme{Copenhagen}
\setbeamertemplate{itemize item}[rectangle] % configure itemize
\setbeamertemplate{itemize subitem}[circle] % configure itemize
\setbeamertemplate{itemize subsubitem}[triangle] % configure itemize
\setbeamertemplate{navigation symbols}{} % remover simbolos de navegacion de las slides
\usefonttheme[onlymath]{serif} % simbolos matematicos en serif (Como es en latex original)

\setbeamertemplate{blocks}[rounded] % blocks corners rounded
\setbeamercolor{block body}{bg=blue!12,fg=black} % color of blocks

% Latex packages
\usepackage{pdfpc-commands} % pdfpc movie commands
\usepackage[utf8]{inputenc}
\usepackage[T1]{fontenc} % para copiar acentos en español del pdf y permite acentos en las notas
\usepackage[spanish]{babel}
\usepackage[binary-units=true,per-mode = symbol]{siunitx} % para manejar las unidades
\usepackage{multirow}
\usepackage{graphicx}
\usepackage{xcolor}
\usepackage{amsmath} % bmatrix
\usepackage[caption=false]{subfig} % caption = false elimina la palabra "Figura" del caption
\setbeamertemplate{caption}{\raggedright\insertcaption\par} % elimina la palabra "Figura" del caption
\usepackage{import} % para el comando import (se usa para pdf_tex)
\captionsetup[subfigure]{labelformat=empty} % remover el indice del caption de la subfigura
\usepackage{booktabs} % \toprule \midrule \bottomrule
\usepackage[overridenote]{pdfpc} % requires to download manually pdfpc.sty package from https://www.ctan.org/pkg/pdfpc
\usepackage[backend=biber]{biblatex} % set biber to format references. Must configure Biber in Texstudio

% Color commands for annotations
\newcommand\TODO[1]{\textbf{\textcolor{red}{#1}}} %  TODO notes

% add bibliography resource
\renewcommand*{\bibfont}{\footnotesize} % change bibliograhy size
\bibliography{./src/bibliography.bib}


% add math preamble
%\usepackage{amsmath}
\usepackage{amssymb}
\usepackage{amsopn}
\usepackage{mathtools}
\usepackage{nicematrix} % Add colors to matrix


% set matrix maximum length
\setcounter{MaxMatrixCols}{20}

% math
\renewcommand{\vec}[1]{\boldsymbol{\mathbf{#1}}}
\newcommand{\norm}[1]{\lVert#1\rVert}

% Declare arg max and arg min functionss
\DeclareMathOperator*{\argmax}{arg\,max}
\DeclareMathOperator*{\argmin}{arg\,min}

% Declare atan2 
\DeclareMathOperator{\atantwo}{atan2}

% Homogeneous decoration function
\newcommand{\homo}[1]{\dot{#1}}


% Declare projection as math function
\DeclareMathOperator{\proj}{proj}
\newcommand{\fromCoord}[2]{{#1}_\mathrm{#2}}
\newcommand{\toCoord}[2]{\prescript{\mathrm{#2}}{}{#1}}
\newcommand{\worldCoordSystem}{\mathrm{W}}
\newcommand{\bodyCoordSystem}{\mathrm{B}}
\newcommand{\cameraCoordSystem}{\mathrm{C}}
\newcommand{\origin}{\vec{o}}
\newcommand{\point}{\vec{p}}
\newcommand{\worldPoint}{\toCoord{\point}{\worldCoordSystem}}
\newcommand{\imagePoint}{\vec{u}}
\newcommand{\cameraPoint}{\toCoord{\point}{\cameraCoordSystem}}
\newcommand{\homoWorldPoint}{\toCoord{\homo{\point}}{\worldCoordSystem}}
\newcommand{\homoImagePoint}{\homo{\imagePoint}}
\newcommand{\homoCameraPoint}{\toCoord{\homo{\point}}{\cameraCoordSystem}}
\newcommand{\measurement}{\vec{z}}
\newcommand{\prediction}{\hat{\vec{z}}}
\newcommand{\seMatrix}{\vec{\xi}}
\newcommand{\transform}[2]{\toCoord{\fromCoord{\seMatrix}{#2}}{#1}}
\newcommand{\pointCoord}[1]{\toCoord{\point}{#1}}
\newcommand{\rotation}{\vec{R}}
\newcommand{\rotationCoord}[2]{\toCoord{\fromCoord{\rotation}{#2}}{#1}}
\newcommand{\translation}{\vec{t}}
\newcommand{\translationCoord}[2]{\toCoord{\fromCoord{\translation}{#2}}{#1}}
\newcommand{\intrinsicMatrix}{\vec{K}}
\newcommand{\principalPoint}{\vec{c}}
\newcommand{\reprojectionError}{u}
\newcommand{\projectionMatrix}{\vec{P}}
\newcommand{\cameraCenter}{\vec{o}}
\newcommand{\worldCameraCenter}{\toCoord{\cameraCenter}{\worldCoordSystem}}
\newcommand{\essentialMatrix}{\vec{E}}
\newcommand{\fundamentalMatrix}{\vec{F}}
\newcommand{\inverse}[1]{{#1}^{-1}}
\newcommand{\epipole}{\vec{e}}

% Localization (State Estimation)
\newcommand{\state}{x}
\newcommand{\observation}{z}
\newcommand{\controlCommand}{u}
\newcommand{\covariance}{\Sigma}
\newcommand{\motionModelNoise}{\epsilon}
\newcommand{\measurementModelNoise}{\delta}
\newcommand{\motionModelFunction}[1]{g\left( #1 \right)}
\newcommand{\observationModelFunction}[1]{h\left( #1 \right)}
\newcommand{\motionParametersCovariance}{R}
\newcommand{\observationModelCovariance}{Q}
\newcommand{\motionModelJacobian}{G}
\newcommand{\observationModelJacobian}{H}
\newcommand{\kalmanGain}{K}
\newcommand{\normalDistribution}[2]{\mathcal{N}\left( {#1}, {#2} \right)}
\newcommand{\motionModelJacobianControl}{V}
\newcommand{\motionModelCovariance}{M}
\newcommand{\stateEvolutionMatrix}{A}

% Mapping slides
\newcommand{\map}{m}
\newcommand{\mapRandomVariable}{m}

% SLAM slides
\newcommand{\informationMatrix}{\vec{\Omega}}
\newcommand{\error}{\vec{e}}
\newcommand{\observationBold}{\vec{z}}
\newcommand{\stateBold}{\vec{x}}
\newcommand{\jacobian}{\vec{J}}
\newcommand{\linearSystemb}{\vec{b}}
\newcommand{\linearSystemH}{\vec{H}}
\newcommand{\covarianceBold}{\vec{\covariance}}


% Motion Planning slides
\newcommand{\workSpace}{\mathcal{W}}
\newcommand{\obstaclesSet}{\mathcal{O}}
\newcommand{\robotInConfiguration}{\mathcal{A}}
\newcommand{\robotConfiguration}{q}
\newcommand{\configurationSpace}{\mathcal{C}}
\newcommand{\freeConfigurationSpace}{\configurationSpace_{free}}
\newcommand{\obstableConfigurationSpace}{\configurationSpace_{obs}}
\newcommand{\goalSet}{\configurationSpace_{goal}}
\newcommand{\startConfiguration}{\robotConfiguration_{I}}
\newcommand{\goalConfiguration}{\robotConfiguration_{G}}
\newcommand{\continuousPath}{\tau}
\newcommand{\motionLaw}{\gamma}
\newcommand{\robotActionSpace}{\mathcal{U}}


% Motion model
\newcommand{\position}{\vec{p}}
\newcommand{\orientation}{\vec{O}}
\newcommand{\orientationQuaternion}{\vec{q}}
\newcommand{\predictedPosition}{\hat{\vec{p}}}
\newcommand{\predictedOrientationQuaternion}{\hat{\vec{q}}}
\newcommand{\linearVelocity}{\vec{v}}
\newcommand{\angularVelocity}{\vec{\omega}}

\DeclareMathOperator{\slerpOp}{slerp}
\newcommand{\slerp}[1]{\slerpOp{\left( #1 \right)}}

% Map structure
\newcommand{\keyframesSet}{K}
\newcommand{\mapPointsSet}{P}
\newcommand{\observedMapPoints}{O}
\newcommand{\covisibilityKeyframes}{CK}
\newcommand{\localMap}{local\_map}

% Bundle Adjutment
\newcommand{\update}{\vec{\delta}}
\newcommand{\incremental}{\hat{\update}}


% Loop Closure names

% scaled operators and letters to fancy view
\newcommand{\sminus}{\scalebox{0.5}[1.0]{$-$}}
\newcommand{\splus}{\scalebox{0.6}[0.6]{$+$}}
\newcommand{\curr}{c}
\newcommand{\sind}[1]{\scalebox{0.6}[0.6]{$#1$}}
\newcommand{\ind}[1]{\scalebox{0.7}[0.7]{$#1$}}

\newcommand{\keyframe}{\vec{K}}
\newcommand{\bowVector}{\vec{v}}
\newcommand{\lcError}{\vec{\Omega}}
\newcommand{\relativeTransformation}{\seMatrix}
\DeclareMathOperator{\interpolate}{interpolate}

\newcommand{\relativeMotion}{\vec{\delta}}
\newcommand{\groundTruth}[1]{{#1}^{*}}

% definición del operador rot()
\DeclareMathOperator{\rotationOp}{rot}
\newcommand{\getRotation}[1]{\rotationOp{\left( #1 \right)}}

\DeclareMathOperator{\translationOp}{trans}
\newcommand{\getTranslation}[1]{\translationOp{\left( #1 \right)}}









\title{Robótica Móvil}
\author{}
\institute{Universidad Nacional de Rosario}
%\date{\scriptsize{Julio 1, 2021}}
\date{}

\begin{document}

% add title page
\frame{\titlepage}


\section{Organización de la Materia}
\begin{frame}
	\frametitle{Programa de la materia (Parte I)}
	\footnotesize
	\begin{enumerate}
		\item {\bf Introducción:} Historia de la Robótica Móvil. Tipos de Robots. Campos de aplicación de la Robótica Móvil. Desafíos de la Robótica Móvil.
		
		\item {\bf Percepción:} tipos de sensores. Sensores interoceptivos y exteroceptivos. Modelo de sensores. Ventajas y desventajas de cada tipo de sensor. Caracterización del ruido.
		
		\item {\bf Cinemática:} Sistemas de locomoción. Modelo Diferencial. Modelo de Ackerman. Holonómico/No-Holonómico.
		
		\item {\bf ROS y Gazebo:} Introducción a ROS 2. Herramientas de visualización y depuración. Recolección de Datos. Simulador Gazebo.
		
		\item {\bf Visión en Robótica:} Geometría proyectiva. Extracción de características visuales. Calibración Visual: intrínseca y Extrínseca. 
		
	\end{enumerate}

\end{frame}

\begin{frame}
	\frametitle{Programa de la materia (Parte II)}
	\footnotesize
	\begin{enumerate}

		\item {\bf Localización:} Modelo probabilístico. Teoría de Bayes. Principio de independencia de Markov. Filtros Gausianos: Filtro Extendido de Kalman. Filtros no-paramétricos: Monte Carlo e Histograma.

		\item {\bf Mapeo:} Nube de puntos. Grilla de Ocupación, árbol cuaternario (Quadtree) y árbol octal (Octree), campos de distancia de signo truncado (TSDF).
		
		\item {\bf Localización y Mapeo Simultáneo (SLAM):} Grafo de Factores. Métodos de optimización: Método de descenso por gradiente, Método Gauss-Newton, Método Levenberg-Marquardt y Bundle Adjustment. Grupos de Lie. Álgebra de Lie. Pre-integración. Problema del Robot Secuestrado (Kidnapped Robot Problem). Relocalización. Detección y Cierre de Ciclos.
		
		\item {\bf Planeamiento de Caminos:} Algoritmo A*. Algoritmo de Dijkstra. Grafo de Visibilidad. Descomposición de celdas. Diagrama de Voronoi. Campos de potencial artificial, Probabilistic RoadMap, Rapidly Exploring Random Tree (RRT) y Rapidly-exploring Random Graph (RRG).
		
		\item {\bf Control:} Controlador proporcional-integral-derivativo (PID). Regulador Lineal Cuadrático (LQR). Control Predictivo por Modelo (MPC).
		
	\end{enumerate}

\end{frame}

\begin{frame}
	\frametitle{Métodología de evaluación}
	
	Correlativas:
	\begin{itemize}
		\item Álgebra Lineal
		\item Probabilidad y Estadística
		\item Complementos de Matemática I
		\item Métodos numéricos
		\item Estructura de Datos y Algoritmos II
		\item Sistemas Operativos I
		\item Introducción a la Inteligencia Artificial
	\end{itemize}
	
	Regularizar:
	\begin{itemize}
		\item Entregas + Trabajos Prácticos (en grupos de 2)
		\item 1 Parcial
	\end{itemize}

	Evaluación final:
	\begin{itemize}
		\item Trabajo Práctico + coloquio (en grupos de 2)
	\end{itemize}	
\end{frame}

\section{Introducción}
\begin{frame}
    \frametitle{Un poco de Historia}
    
    La primera aparición de la palabra robot es utilizada por Karl Capek en 1921 en su obra teatral R. U. R. (Rossum's Universal Robots). La palabra robot viene de la palabra checa \emph{robota} que significa esclavo.
    
    \TODO{agregar imagen obra de teatro wikipedia.}
    
    \note{
        Fuente: https://cs.stanford.edu/people/eroberts/courses/soco/projects/1998-99/robotics/history.html\\
        La obra teatral trata sobre una empresa que construye humanos artificiales orgánicos con el fin de aligerar la carga de trabajo del resto de personas. Aunque en la obra a estos hombres artificiales se les llama robots, tienen más que ver con el concepto moderno de androide o clon. Se trata de criaturas que pueden hacerse pasar por humanos y que tienen el don de poder pensar. Pese a ser creadas para ayudar a la humanidad, más adelante estas máquinas entrarán en confrontación con la sociedad, iniciando una revolución que acabará destruyendo la humanidad.}
\end{frame}

\begin{frame}
    \frametitle{Un poco de Historia}
    
    La palabra \emph{robotics} también fue utilizada por primera vez por Issac Asimov en 1942 en su cuento corto \emph{Runaround} (en español Circuito Vicioso).
    
    Primera Ley
    Un robot no hará daño a un ser humano ni, por inacción, permitirá que un ser humano sufra daño.
    Segunda Ley
    Un robot debe cumplir las órdenes dadas por los seres humanos, a excepción de aquellas que entren en conflicto con la primera ley.
    Tercera Ley
    Un robot debe proteger su propia existencia en la medida en que esta protección no entre en conflicto con la primera o con la segunda ley.
    
    \note{The word robotics was also coined by a writer.  Russian-born American science-fiction writer Isaac Asimov first used the word in 1942 in his short story Runabout.  Asimov had a much brighter and more optimistic opinion of the robot's role in human society than did Capek.  He generally characterized the robots in his short stories as helpful servants of man and viewed robots as a better, cleaner race.  Asimov also proposed three "Laws of Robotics" that his robots, as well as sci-fi robotic characters of many other stories.}
\end{frame}

\begin{frame}
    \frametitle{¿Qué es un Robot?}
    \note{hay varias definiciones.\\
        Fuente: https://robots.ieee.org/learn/what-is-a-robot/}
    
    Según Robot Institute of America, 1979
    Un manipulador reprogramable y multifuncional diseñado para mover material, piezas, herramientas o dispositivos especializados a través de varios movimientos programados para el desempeño de una variedad de tareas.
    
    Según la RAE
    Máquina o ingenio electrónico programable que es capaz de manipular objetos y realizar diversas operaciones.
    
    Una definición que usamos (https://robots.ieee.org/learn/what-is-a-robot/):
    Un robot es una máquina autónoma capaz de sensar su entorno, realizar cálculos para tomar decisiones y realizar acciones en el mundo real.
    
    \note{}
\end{frame}

\begin{frame}
    \frametitle{¿Qué es la Robótica?}
    La robótica es la intersección de la ciencia, la ingeniería y la tecnología que produce máquinas, llamadas robots, que sustituyen (o replican) las acciones humanas.
\end{frame}

\begin{frame}
    \frametitle{Aplicaciones de la Robótica}
%    \begin{figure}[!h]
%        \centering
%        \subfloat[] 
%        {
%            \includegraphics[width=0.33\columnwidth]{./images/introduction/drone.jpg}
%        }
%        \subfloat[] 
%        {
%            \includegraphics[width=0.33\columnwidth]{./images/introduction/GoogleCar.jpg}
%        }
%        \subfloat[] 
%        {
%            \includegraphics[width=0.16\columnwidth]{./images/introduction/industrial_robot.png}
%        }\\
%        \subfloat[] 
%        {
%            \includegraphics[width=0.33\columnwidth]{./images/introduction/curiosity.png}
%        }
%        \subfloat[] 
%        {
%            \includegraphics[width=0.33\columnwidth]{./images/introduction/hortibot.png}
%        }
%        \subfloat[]
%        {
%            \includegraphics[width=0.33\columnwidth]{./images/introduction/aqua2.png}
%        }		 
%    \end{figure}
\end{frame}

\begin{frame}
    \frametitle{Navegación autónoma}
    \begin{block}{}
        La navegación autónoma puede definirse a grandes rasgos como la capacidad de moverse de forma segura a lo largo de una trayectoria entre un punto de inicio y uno final [1].
    \end{block}
    \vspace{5mm}
    \begin{columns}
        \column{0.4\textwidth}
        \hspace{13pt}Pregunta:
        \begin{enumerate}
            \visible<2-7>{ \item[-] ¿Dónde estoy?}
            \visible<4-7>{\item[-] ¿Por dónde estoy yendo?}
            \visible<6-7>{\item[-] ¿Cómo llego hasta allí?}
        \end{enumerate}
        \column{0.6\textwidth}
        Respuesta:
        \begin{enumerate}[$\rightarrow$]
            \visible<3-7>{ \item  Cálculo de la posición (Localization)}
            \visible<5-7>{ \item  Representación del entorno (Mapping)}
            \visible<7-7>{\item  Planeamiento de movimiento (Motion planning)}
        \end{enumerate}
    \end{columns}
    \only<4>{}
    \vfill
    \begin{tiny}
        [1] J. J. Leonard - et al., ``Mobile robot localization by ...,'' IEEE Transactions on Robotics and Automation, 2002.
    \end{tiny}
\end{frame}

\section{Trabajo actual}
\begin{frame}
    \frametitle{Trabajo actual}
    \begin{itemize}
        \item Fusión de sensores (Loosely y Tightly coupled)
        \item SLAM Relativo (SLAM Multi-Robot y Distribuido)
        \item Calibración de sensores
        \item Long-term/Loop Closing en entornos Agrícolas
    \end{itemize}
\end{frame}

%\section*{Referencias}
%\input{src/references.tex}

\end{document}
