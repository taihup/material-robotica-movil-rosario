\documentclass[tp]{lcc}

% add latex preamble
% para la bibliografía se requiere biber y configurar texstudio

% Latex packages
\usepackage[utf8]{inputenc}
\usepackage[T1]{fontenc} % para copiar acentos en español del pdf y permite acentos en las notas
\usepackage[spanish]{babel}
\usepackage[per-mode = symbol]{siunitx} % para manejar las unidades
\usepackage{multimedia} % to add videos with \movie command
\usepackage{multirow}
\usepackage{graphicx}
\usepackage{xcolor}
\usepackage{amsmath} % bmatrix
\usepackage[makeroom]{cancel} % \cancel to cancel terms in math equations
\renewcommand{\CancelColor}{\color{red}} % set red color for \cancel command
\usepackage[caption=false]{subfig} % caption = false elimina la palabra "Figura" del caption
\usepackage{import} % para el comando import (se usa para pdf_tex)
\captionsetup[subfigure]{labelformat=empty} % remover el indice del caption de la subfigura
\usepackage{booktabs} % \toprule \midrule \bottomrule
\usepackage[backend=biber]{biblatex} % set biber to format references. Must configure Biber in Texstudio
\usepackage{csquotes} % to remove warning triggered by biblatex and babel
\usepackage{algorithm} % to put captions to the algorithmics environmets
\usepackage{algpseudocode} % to write algorithm
\usepackage{tikz} % to use tikz
\usetikzlibrary{fit} % to fit a node around other nodes in tikz
\usepackage[export]{adjustbox} % valign in subfloat
\usepackage{colortbl} % to paint cells in a table

% Color commands for annotations
\newcommand\TODO[1]{\textbf{\textcolor{red}{#1}}} %  TODO notes

% Graphic paths
\graphicspath{{./images/}}

% listings configuration for C code
\usepackage{listings} % code
\definecolor{commentgreen}{RGB}{2,112,10}
\definecolor{eminence}{RGB}{108,48,130}
\definecolor{weborange}{RGB}{255,165,0}
\definecolor{frenchplum}{RGB}{129,20,83}

\lstset{ % spanish characters for listings package
	inputencoding=latin1,
    columns=fullflexible,
	breaklines=true,
	tabsize=2,
	showstringspaces=false,
	basicstyle=\ttfamily,
	backgroundcolor=\color{lightgray}, % Choose background color
	literate={á}{{\'a}}1
	{ã}{{\~a}}1
	{é}{{\'e}}1
	{ó}{{\'o}}1
	{í}{{\'i}}1
	{ñ}{{\~n}}1
	{¡}{{!`}}1
	{¿}{{?`}}1
	{ú}{{\'u}}1
	{Í}{{\'I}}1
	{Ó}{{\'O}}1
    {-}{-}1
}

\lstdefinestyle{cpp}{ % spanish characters for listings package
    language=C++,
   	commentstyle=\color{commentgreen},
    keywordstyle=\color{eminence},
    stringstyle=\color{red},
    emph={int,char,double,float,unsigned,void,bool},
    emphstyle={\color{blue}}
}

\lstdefinestyle{bash}{ % spanish characters for listings package
	language=Bash
}

\lstdefinestyle{xml}{
	language=XML,
	morekeywords={encoding,xs:schema,xs:element,xs:complexType,xs:sequence,xs:attribute}
}

\lstdefinestyle{cmake}{
	language=make, % there is no cmake support in listings
}

\lstdefinestyle{python}{
    language=python,
}


%%%%% PARA QUE EN LAS TABLAS SE PUEDA PONER UN SALTO DE LINEA DENTRO DE UNA CELDA
\newcommand{\specialcell}[2][c]{%
    \begin{tiny}
        \begin{tabular}[#1]{@{}c@{}}#2\end{tabular}  
    \end{tiny}
}
%%%%%%%%%%%%%%%%%%%%%%%%%%%%%%%%%%%%%%%%%%%%%%%%%%%%%%%%%%%%%%%%%%%%%%%%

%%%%% PARA QUE LAS TABLAS TENGAN TODAS LAS COLUMNAS CENTRADAS Y DE IGUAL TAMAÑO
\usepackage{tabularx}
\renewcommand{\tabularxcolumn}[1]{>{\centering\arraybackslash}m{#1}}
%%%%%%%%%%%%%%%%%%%%%%%%%%%%%%%%%%%%%%%%%%%%%%%%%%%%%%%%%%%%%%%%%%%%%%%%



% add math preamble
\usepackage{amsmath}
\usepackage{amssymb}
\usepackage{amsopn}
\usepackage{mathtools}

% math
\renewcommand{\vec}[1]{\boldsymbol{\mathbf{#1}}}
\newcommand{\norm}[1]{\lVert#1\rVert}

% Declare arg max and arg min functionss
\DeclareMathOperator*{\argmax}{arg\,max}
\DeclareMathOperator*{\argmin}{arg\,min}

% Homogeneous decoration function
\newcommand{\homo}[1]{\dot{#1}}


% Declare projection as math function
\DeclareMathOperator{\proj}{proj}
\newcommand{\fromCoord}[2]{{#1}_\mathrm{#2}}
\newcommand{\toCoord}[2]{\prescript{\mathrm{#2}}{}{#1}}
\newcommand{\worldCoordSystem}{\mathrm{w}}
\newcommand{\bodyCoordSystem}{\mathrm{B}}
\newcommand{\cameraCoordSystem}{\mathrm{c}}
\newcommand{\point}{\vec{p}}
\newcommand{\worldPoint}{\toCoord{\point}{\worldCoordSystem}}
\newcommand{\imagePoint}{\vec{u}}
\newcommand{\cameraPoint}{\toCoord{\point}{\cameraCoordSystem}}
\newcommand{\homoWorldPoint}{\toCoord{\homo{\point}}{\worldCoordSystem}}
\newcommand{\homoImagePoint}{\homo{\imagePoint}}
\newcommand{\homoCameraPoint}{\toCoord{\homo{\point}}{\cameraCoordSystem}}
\newcommand{\measurement}{\vec{z}}
\newcommand{\prediction}{\hat{\vec{z}}}
\newcommand{\seMatrix}{\vec{\xi}}
\newcommand{\transform}[2]{\toCoord{\fromCoord{\seMatrix}{#2}}{#1}}
\newcommand{\pointCoord}[1]{\toCoord{\point}{#1}}
\newcommand{\rotation}{\vec{R}}
\newcommand{\rotationCoord}[2]{\toCoord{\fromCoord{\rotation}{#2}}{#1}}
\newcommand{\translation}{\vec{t}}
\newcommand{\translationCoord}[2]{\toCoord{\fromCoord{\translation}{#2}}{#1}}
\newcommand{\intrinsicMatrix}{\vec{K}}
\newcommand{\principalPoint}{\vec{c}}
\newcommand{\reprojectionError}{u}
\newcommand{\projectionMatrix}{\vec{P}}
\newcommand{\cameraCenter}{\vec{o}}
\newcommand{\essentialMatrix}{\vec{E}}
\newcommand{\inverse}[1]{{#1}^{-1}}

% Motion model
\newcommand{\position}{\vec{p}}
\newcommand{\orientationQuaternion}{\vec{q}}
\newcommand{\predictedPosition}{\hat{\vec{p}}}
\newcommand{\predictedOrientationQuaternion}{\hat{\vec{q}}}
\newcommand{\linearVelocity}{\vec{v}}
\newcommand{\angularVelocity}{\vec{\omega}}

\DeclareMathOperator{\slerpOp}{slerp}
\newcommand{\slerp}[1]{\slerpOp{\left( #1 \right)}}

% Map structure
\newcommand{\map}{M}
\newcommand{\keyframesSet}{K}
\newcommand{\mapPointsSet}{P}
\newcommand{\observedMapPoints}{O}
\newcommand{\covisibilityKeyframes}{CK}
\newcommand{\localMap}{local\_map}



% Bundle Adjutment
\newcommand{\update}{\vec{\delta}}
\newcommand{\incremental}{\hat{\update}}


% Loop Closure names

% scaled operators and letters to fancy view
\newcommand{\sminus}{\scalebox{0.5}[1.0]{$-$}}
\newcommand{\splus}{\scalebox{0.6}[0.6]{$+$}}
\newcommand{\curr}{c}
\newcommand{\sind}[1]{\scalebox{0.6}[0.6]{$#1$}}
\newcommand{\ind}[1]{\scalebox{0.7}[0.7]{$#1$}}

\newcommand{\keyframe}{\vec{K}}
\newcommand{\bowVector}{\vec{v}}
\newcommand{\lcError}{\vec{\Omega}}
\newcommand{\relativeTransformation}{\seMatrix}
\DeclareMathOperator{\interpolate}{interpolate}

\newcommand{\relativeMotion}{\vec{\delta}}
\newcommand{\groundTruth}[1]{{#1}^{*}}



% definición del operador rot()
\DeclareMathOperator{\rotationOp}{rot}
\newcommand{\getRotation}[1]{\rotationOp{\left( #1 \right)}}

\DeclareMathOperator{\translationOp}{trans}
\newcommand{\getTranslation}[1]{\translationOp{\left( #1 \right)}}









\codigo{R-521}
\materia{Robótica Móvil}
\titulo{Visión por Computadora: Reconstrucción 3D y Estimación de Pose}

\soluciones
\commentstrue


\usepackage{biblatex}
%\addbibresource{refs.bib}

\begin{document}
\maketitle

\section{Introducción}

El objetivo del trabajo práctico es la realización de los pasos básicos para poder triangular y proyectar puntos con una cámara estéreo. En este trabajo se debe utilizar  las librerías OpenCV\footnote{\url{https://opencv.org/}} y software de calibración ampliamente utilizado en el campo de visión por computadora y robótica.


\section{Entrega}
\begin{itemize}
	\item Se debe proveer un repositorio git que contenga el código desarrollado y un archivo \lstinline{README.md} con las instrucciones de compilación y ejecución. Se recomienda hacer una imagen Docker para facilitar la reproducción de los resuldatos.
    
	\item Se debe entregar un informe en Lyx o \LaTeX\  explicando el trabajo realizado y analizando los resultados obtenidos.
\end{itemize}

\section{Evaluación}
\begin{itemize}
    \item Haciendo únicamente los ejercicios obligarorios (no opcionales), el trabajo tiene una nota máxima de 8. Los ejercicios opcionales permiten llegar a la nota máxima de 10.
    \item Entregas Fuera de Término: Si la entrega se realiza durante la primer semana luego del plazo, se decuentan 2 puntos. Si la entrega es más tarde que esto último la nota máxima es de 6.
\end{itemize}

\section{Datos}
Utilizar el dataset EuRoC\footnote{\url{https://projects.asl.ethz.ch/datasets/doku.php?id=kmavvisualinertialdatasets}}. Debera convertir la secuencia de calibración a formato ROS2. puede utilizar el software \lstinline{rosbags}\footnote{https://pypi.org/project/rosbags/}. El resto de las secuencias ya se encuentran en formato ROS2 y pueden ser descargadas de \url{https://docs.openvins.com/gs-datasets.html}


\section{Ejercicios}

\ejercicio Realizar una calibración de la cámara estéreo del robot. Se debe proveer los parámetros intrínsecos y extrínsecos.

\begin{nota}
    Para la calibración se puede utilizar cualquiera de las aplicaciones que se detallan a continuación:
    \begin{itemize}
        \item ROS2 camera\_calibration:  \url{https://navigation.ros.org/tutorials/docs/camera_calibration.html}
        \item OpenCV tutorial\_camera\_calibration:\\ \url{https://docs.opencv.org/4.x/d4/d94/tutorial_camera_calibration.html}
        \item Kalibr: \url{https://github.com/ethz-asl/kalibr/wiki/ROS2-Calibration-Using-Kalibr}
        \item Basal Calibration: \url{https://github.com/VladyslavUsenko/basalt/blob/master/doc/Calibration.md}
        \item BoofCV: \url{https://boofcv.org/index.php?title=Tutorial_Camera_Calibration}
        \item Camera Calibration Toolbox for Matlab:\\ \url{https://www.cs.toronto.edu/pub/psala/VM/cameraCalibrationExample.html}
    \end{itemize}
\end{nota}

\ejercicio Desarrollar un programa que lea un par de imágenes estéreo cualquiera y realice los siguientes pasos:

\begin{enumerate}
    
    %\item \textbf{Sincronizar imágenes}
    %Debido a que en el rosbag las imágenes izquierda y derecha se almacenaron en un instante (timestamp en el rosbag) distinto al que fueron capturadas (timestamp en el header de las imágenes). Vamos a tener que sincronizar las imágenes utilizando el timestamp que figura en el header, esto se hace mediante el paquete \lstinline{message_filters}\footnote{\url{http://wiki.ros.org/message_filters}}. Para esto hay que 
    %
    %\begin{lstlisting}[style=python]
    %# Importamos del paquete
    %import message_filters
    %\end{lstlisting}
    %
    %\begin{lstlisting}[style=python] 
    %# Creamos subscribers a las cámaras izq y der
    %self.left_rect = message_filters.Subscriber(self, Image, '/left/image_rect')
    %self.right_rect = message_filters.Subscriber(self, Image, '/right/image_rect')
    %\end{lstlisting}
    %
    %\begin{lstlisting}[style=python] 
    %# Creamos el objeto ts del tipo TimeSynchronizer encargado de sincronizar los mensajes recibidos.
    %ts = message_filters.TimeSynchronizer([self.left_rect, self.right_rect], 10)
    %# Registramos un callback para procesar los mensajes sincronizados.
    %ts.registerCallback(self.callback)
    %La función callback es un método del nodo con esta asignatura:
    %def callback(self, left_msg, right_msg):
    %\end{lstlisting}
    
    \item \textbf{Rectificar Imágenes}
    
    Rectificar las imágenes utilizando los parámetros íntrinsecos y extrínsecos de la cámara estéreo. Esto se puede realizar utilizando:
    
    \begin{itemize}
        \item ROS2, por medio del paquete \lstinline{stereo_image_proc}
        
\begin{lstlisting}[style=bash]     
ros2 launch stereo_image_proc stereo_image_proc.launch.py
\end{lstlisting}
        
        Para reproducir el rosbag y remapear los tópicos a los que el paquete \lstinline{stereo_image_proc} requiere puede hacer:
\begin{lstlisting}[style=bash]   
ros2 bag play CARPETA_ROSBAG --remap /stereo/left/image_raw:=/left/image_raw \
/stereo/left/camera_info:=/left/camera_info \
/stereo/right/image_raw:=/right/image_raw \
/stereo/right/camera_info:=/right/camera_info
\end{lstlisting}
        
        
        \item o bien, OpenCV, por medio de las funciones: \lstinline{cv::stereoRectify()},  \lstinline{cv::initUndistortRectifyMap()} y  \lstinline{remap()}.
        
    \end{itemize}

\item \textbf{Extraer Feaures Visuales: Keypoints y descriptores}

Selecionar un detector de keypoints (FAST, ORB, SIFT, SURF, GFTT, BRISK, etc.) y un descriptor (BRIEF, ORB, BRISK, etc.), y extraer features en ambas imágenes. Capturar una imagen izquierda y derecha con los features extraídos. Agregar captura al informe.

\item \textbf{Buscar correspondencias visuales}

Realizar la búsqueda de correspondencias entre los feature de ambas imágenes (\emph{matching}). Para esto se debe utilizar la función \lstinline{cv::BFMatcher::BFMatcher()}. Visualizar todos los matches. Luego, visualizar todos los matches que tengan una distancia de matching menor a 30. Agregar capturas al informe.

\item \textbf{Triangular Puntos 3D}

Dadas las correspondencias visuales (\emph{matches}) obtenidas en el paso anterior, realizar la triangulación de los features extraídos utilizando la función \lstinline{cv::sfm::triangulatePoints()}. Para la visualización de la nube de puntos 3D se puede publicar un mensaje de tipo \lstinline{sensor_msgs/PointCloud2}\footnote{\url{http://docs.ros.org/en/melodic/api/sensor_msgs/html/msg/PointCloud2.html}} y hacer uso de RViz. Agregar capturas al informe.

\item \textbf{Filtrar de Correspondencias Espúreas}

Aplicar RANSAC (\emph{Random sample consensus}) para filtrar los matches espúreos y computar la Matríz homográfica que relaciona los puntos. Para esto puede utilizar la función \lstinline{cv::findHomography()}. Para verificar el impacto del filtrado, visualizar los matches entre las imágenes nuevamente como en la nube de puntos 3D generada. También, visualizar en la imagen derecha los puntos de la imagen izquierda transformados por la matríz homográfica. Para esto último utilizar la función  \lstinline{cv::perspectiveTransform()}. Agregar capturas al informe.

\item (Opcional) \textbf{Feature Mapping con Localización \emph{ground-truth}}

Mapear el entorno utilizando las poses dada por el \emph{ground-truth} del dataset. Visualizar el mapa reconstruido. Agregar captura al informe.

\item \textbf{Computar el Mapa de Disparidad}

Computar el mapa de disparidad con las librerías utilizando la función \lstinline{cv::StereoMatcher::compute()}. Opcionalmente para tener mejores resultados puede utilizar la librería LIBELAS\footnote{\url{http://www.cvlibs.net/software/libelas/}}. Visualizar el mapa de disparidad. Agregar captura al informe.

\item \textbf{Reconstruir la Escena 3D de manera Densa}

Utilizando el mapa de disparidad obtenido en el paso anterior realizar una reconstrucción densa de la escena observada utilizando la función \lstinline{cv::reprojectImageTo3D()}. Para esto debe utilizar la matríz de reproyección Q retornada por la función \lstinline{cv::stereoRectify()}. Visualizar la nube de puntos 3D. Agregar captura al informe.

\item (Opcional) \textbf{Dense Mapping con Localización Ground-truth}

Mapear el entorno de manera densa utilizando las poses dada por el \emph{ground-truth} del dataset. Visualizar el mapa reconstruido. Agregar captura al informe.

\item \textbf{Estimar la Pose utilizando Visión Monocular}

Utilizando \lstinline{cv::recoverPose()} estimar la transformación entre la cámara izquierda y la cámara derecha. Para esto deberá calcular la matríz esencial utilizando la función \lstinline{cv::findEssentialMat()}. Notar que \lstinline{cv::recoverPose()} retorna el vector unitario de traslación, por lo tanto deberá multiplicarlo por el factor de escala (en este caso el baseline entre las cámaras) para obtener la traslación estimada. Una vez hecho esto se pide:

\begin{itemize}
    \item Visualizar la pose estimada de ambas cámaras. Agregar captura al informe.
    \item Estimar y visualizar la trayectoría realizada por la cámara izquierda utilizando como factor de escala la distancia entre cada par de frames dada por el \emph{ground-truth}. Agregar captura al informe.
\end{itemize}


\end{enumerate}

\end{document}
