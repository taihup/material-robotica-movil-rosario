\documentclass[tp]{lcc}

% add latex preamble
% para la bibliografía se requiere biber y configurar texstudio

% Latex packages
\usepackage[utf8]{inputenc}
\usepackage[T1]{fontenc} % para copiar acentos en español del pdf y permite acentos en las notas
\usepackage[spanish]{babel}
\usepackage[per-mode = symbol]{siunitx} % para manejar las unidades
\usepackage{multimedia} % to add videos with \movie command
\usepackage{multirow}
\usepackage{graphicx}
\usepackage{xcolor}
\usepackage{amsmath} % bmatrix
\usepackage[makeroom]{cancel} % \cancel to cancel terms in math equations
\renewcommand{\CancelColor}{\color{red}} % set red color for \cancel command
\usepackage[caption=false]{subfig} % caption = false elimina la palabra "Figura" del caption
\usepackage{import} % para el comando import (se usa para pdf_tex)
\captionsetup[subfigure]{labelformat=empty} % remover el indice del caption de la subfigura
\usepackage{booktabs} % \toprule \midrule \bottomrule
\usepackage[backend=biber]{biblatex} % set biber to format references. Must configure Biber in Texstudio
\usepackage{csquotes} % to remove warning triggered by biblatex and babel
\usepackage{algorithm} % to put captions to the algorithmics environmets
\usepackage{algpseudocode} % to write algorithm
\usepackage{tikz} % to use tikz
\usetikzlibrary{fit} % to fit a node around other nodes in tikz
\usepackage[export]{adjustbox} % valign in subfloat
\usepackage{colortbl} % to paint cells in a table

% Color commands for annotations
\newcommand\TODO[1]{\textbf{\textcolor{red}{#1}}} %  TODO notes

% Graphic paths
\graphicspath{{./images/}}

% listings configuration for C code
\usepackage{listings} % code
\definecolor{commentgreen}{RGB}{2,112,10}
\definecolor{eminence}{RGB}{108,48,130}
\definecolor{weborange}{RGB}{255,165,0}
\definecolor{frenchplum}{RGB}{129,20,83}

\lstset{ % spanish characters for listings package
	inputencoding=latin1,
    columns=fullflexible,
	breaklines=true,
	tabsize=2,
	showstringspaces=false,
	basicstyle=\ttfamily,
	backgroundcolor=\color{lightgray}, % Choose background color
	literate={á}{{\'a}}1
	{ã}{{\~a}}1
	{é}{{\'e}}1
	{ó}{{\'o}}1
	{í}{{\'i}}1
	{ñ}{{\~n}}1
	{¡}{{!`}}1
	{¿}{{?`}}1
	{ú}{{\'u}}1
	{Í}{{\'I}}1
	{Ó}{{\'O}}1
    {-}{-}1
}

\lstdefinestyle{cpp}{ % spanish characters for listings package
    language=C++,
   	commentstyle=\color{commentgreen},
    keywordstyle=\color{eminence},
    stringstyle=\color{red},
    emph={int,char,double,float,unsigned,void,bool},
    emphstyle={\color{blue}}
}

\lstdefinestyle{bash}{ % spanish characters for listings package
	language=Bash
}

\lstdefinestyle{xml}{
	language=XML,
	morekeywords={encoding,xs:schema,xs:element,xs:complexType,xs:sequence,xs:attribute}
}

\lstdefinestyle{cmake}{
	language=make, % there is no cmake support in listings
}

\lstdefinestyle{python}{
    language=python,
}


%%%%% PARA QUE EN LAS TABLAS SE PUEDA PONER UN SALTO DE LINEA DENTRO DE UNA CELDA
\newcommand{\specialcell}[2][c]{%
    \begin{tiny}
        \begin{tabular}[#1]{@{}c@{}}#2\end{tabular}  
    \end{tiny}
}
%%%%%%%%%%%%%%%%%%%%%%%%%%%%%%%%%%%%%%%%%%%%%%%%%%%%%%%%%%%%%%%%%%%%%%%%

%%%%% PARA QUE LAS TABLAS TENGAN TODAS LAS COLUMNAS CENTRADAS Y DE IGUAL TAMAÑO
\usepackage{tabularx}
\renewcommand{\tabularxcolumn}[1]{>{\centering\arraybackslash}m{#1}}
%%%%%%%%%%%%%%%%%%%%%%%%%%%%%%%%%%%%%%%%%%%%%%%%%%%%%%%%%%%%%%%%%%%%%%%%



% add math preamble
\usepackage{amsmath}
\usepackage{amssymb}
\usepackage{amsopn}
\usepackage{mathtools}

% math
\renewcommand{\vec}[1]{\boldsymbol{\mathbf{#1}}}
\newcommand{\norm}[1]{\lVert#1\rVert}

% Declare arg max and arg min functionss
\DeclareMathOperator*{\argmax}{arg\,max}
\DeclareMathOperator*{\argmin}{arg\,min}

% Homogeneous decoration function
\newcommand{\homo}[1]{\dot{#1}}


% Declare projection as math function
\DeclareMathOperator{\proj}{proj}
\newcommand{\fromCoord}[2]{{#1}_\mathrm{#2}}
\newcommand{\toCoord}[2]{\prescript{\mathrm{#2}}{}{#1}}
\newcommand{\worldCoordSystem}{\mathrm{w}}
\newcommand{\bodyCoordSystem}{\mathrm{B}}
\newcommand{\cameraCoordSystem}{\mathrm{c}}
\newcommand{\point}{\vec{p}}
\newcommand{\worldPoint}{\toCoord{\point}{\worldCoordSystem}}
\newcommand{\imagePoint}{\vec{u}}
\newcommand{\cameraPoint}{\toCoord{\point}{\cameraCoordSystem}}
\newcommand{\homoWorldPoint}{\toCoord{\homo{\point}}{\worldCoordSystem}}
\newcommand{\homoImagePoint}{\homo{\imagePoint}}
\newcommand{\homoCameraPoint}{\toCoord{\homo{\point}}{\cameraCoordSystem}}
\newcommand{\measurement}{\vec{z}}
\newcommand{\prediction}{\hat{\vec{z}}}
\newcommand{\seMatrix}{\vec{\xi}}
\newcommand{\transform}[2]{\toCoord{\fromCoord{\seMatrix}{#2}}{#1}}
\newcommand{\pointCoord}[1]{\toCoord{\point}{#1}}
\newcommand{\rotation}{\vec{R}}
\newcommand{\rotationCoord}[2]{\toCoord{\fromCoord{\rotation}{#2}}{#1}}
\newcommand{\translation}{\vec{t}}
\newcommand{\translationCoord}[2]{\toCoord{\fromCoord{\translation}{#2}}{#1}}
\newcommand{\intrinsicMatrix}{\vec{K}}
\newcommand{\principalPoint}{\vec{c}}
\newcommand{\reprojectionError}{u}
\newcommand{\projectionMatrix}{\vec{P}}
\newcommand{\cameraCenter}{\vec{o}}
\newcommand{\essentialMatrix}{\vec{E}}
\newcommand{\inverse}[1]{{#1}^{-1}}

% Motion model
\newcommand{\position}{\vec{p}}
\newcommand{\orientationQuaternion}{\vec{q}}
\newcommand{\predictedPosition}{\hat{\vec{p}}}
\newcommand{\predictedOrientationQuaternion}{\hat{\vec{q}}}
\newcommand{\linearVelocity}{\vec{v}}
\newcommand{\angularVelocity}{\vec{\omega}}

\DeclareMathOperator{\slerpOp}{slerp}
\newcommand{\slerp}[1]{\slerpOp{\left( #1 \right)}}

% Map structure
\newcommand{\map}{M}
\newcommand{\keyframesSet}{K}
\newcommand{\mapPointsSet}{P}
\newcommand{\observedMapPoints}{O}
\newcommand{\covisibilityKeyframes}{CK}
\newcommand{\localMap}{local\_map}



% Bundle Adjutment
\newcommand{\update}{\vec{\delta}}
\newcommand{\incremental}{\hat{\update}}


% Loop Closure names

% scaled operators and letters to fancy view
\newcommand{\sminus}{\scalebox{0.5}[1.0]{$-$}}
\newcommand{\splus}{\scalebox{0.6}[0.6]{$+$}}
\newcommand{\curr}{c}
\newcommand{\sind}[1]{\scalebox{0.6}[0.6]{$#1$}}
\newcommand{\ind}[1]{\scalebox{0.7}[0.7]{$#1$}}

\newcommand{\keyframe}{\vec{K}}
\newcommand{\bowVector}{\vec{v}}
\newcommand{\lcError}{\vec{\Omega}}
\newcommand{\relativeTransformation}{\seMatrix}
\DeclareMathOperator{\interpolate}{interpolate}

\newcommand{\relativeMotion}{\vec{\delta}}
\newcommand{\groundTruth}[1]{{#1}^{*}}



% definición del operador rot()
\DeclareMathOperator{\rotationOp}{rot}
\newcommand{\getRotation}[1]{\rotationOp{\left( #1 \right)}}

\DeclareMathOperator{\translationOp}{trans}
\newcommand{\getTranslation}[1]{\translationOp{\left( #1 \right)}}









% add bibliography resource
\renewcommand*{\bibfont}{\footnotesize} % change bibliograhy size
\bibliography{../../common/bibliography.bib}

\codigo{R-521}
\materia{Robótica Móvil}
\title{CSE 571 - Robotics \\ Homework 2 - EKF and RRT}
\author{}
\date{}

\usepackage{biblatex}

\begin{document}

\maketitle

\section{Rapidly-exploring Random Tree}
In this part, you are provided with a robot arm that has 2 links is able to move in a 2D plane only. Your task is to implement Rapidly-exploring Random Tree (RRT), a classical sampling-based motion planning algorithms named after its data structure, and study the parameters that govern its behaviors.

\subsection{Code Overview}
The starter code for RRT is under the \texttt{rrt} folder. Here is a brief overview.

\subsection{RRT Implementation}
Implement a RRT planner for the 2D robot in \texttt{RRTPlanner.py} by filling in \texttt{Plan} and \texttt{extend} functions. Your results from a successful RRT implementation should be comparable to the following results:

\begin{verbatim}
$ python plan.py -o 0 --seed 0
...
cost: 198.84383834015168

$ python plan.py -o 2 --seed 0
...
cost: 170.48992200228264
\end{verbatim}

If you turn on the \texttt{-v} flag, i.e. \texttt{python plan.py --seed 0 -v}, you should see a plot similar to Fig. \ref{fig:rrt}.

 \begin{figure}[h]
     \centering
     \includegraphics[width=0.6\textwidth]{example-image-a}
     \caption{Example RRT path in configuration space.}
     \label{fig:rrt}
 \end{figure}

\textbf{Answer the following questions.}

Note that since RRT is non-deterministic, you will need to provide statistical results, i.e. mean and standard deviation over \textbf{at least 5 runs} with different random seeds specified by \texttt{--seed}.

The \texttt{-o} flag specifies the number of obstacles in the environment. Please report your results with \texttt{-o 2}. You can use other values for debugging.

\begin{enumerate}
    \item Bias the sampling to pick the goal with $5\%$, $20\%$ probability. Report the performance (cost and time). For each setting, include at least one figure like Fig. \ref{fig:rrt} showing the RRT tree in configuration space. Which value would you pick in practice?
    \item You can view the robot as a point that can move in arbitrarily directions in the configuration space. In other words, the states can be interpolated via a straight line (see Fig. \ref{fig:extend} for an illustration). Compare two strategies for the \texttt{extend()} function:
    \begin{itemize}
        \item the nearest neighbor extends only half way to the sampled point (i.e. step size $\eta=0.5$)
        \item the nearest neighbor extends all the way till the sampled point (i.e. step size $\eta=1$)
    \end{itemize}
    Report the performance (cost, time) for each setting. Include at least one figure showing the final state of the RRT tree. Which strategy would you employ in practice?
    \item Discuss any challenges you faced and describe your debugging process.
\end{enumerate}

\begin{figure}[h]
    \centering
    \includegraphics[width=0.5\textwidth]{example-image-a}
    \caption{Visualization of \texttt{extend()} with step size $\eta$, which controls the ratio of the distance from $x_{\text{new}}$ (the state to be added) to $x_{\text{near}}$ and the distance from $x_{\text{rand}}$ (the sampled state) to $x_{\text{near}}$. If $\eta=1$, $x_{\text{new}}=x_{\text{rand}}$.}
    \label{fig:extend}
\end{figure}

\textbf{Hint:}
Check out these useful functions that you should use to simplify your implementation:
\begin{itemize}
    \item In \texttt{RRTTree.py}:
    \begin{itemize}
        \item \texttt{AddVertex}
        \item \texttt{AddEdge}
        \item \texttt{GetNearestVertex}
    \end{itemize}
    \item In \texttt{ArmEnvironment.py}:
    \begin{itemize}
        \item \texttt{compute\_distance}
        \item \texttt{goal\_criterion}
        \item \texttt{edge\_validity\_checker}
    \end{itemize}
\end{itemize}

\section{Submission}
We will be using the Canvas for submission of the assignments. Please submit the written assignment answers as a PDF. For the code, submit a zip file of the entire working directory.

\printbibliography

\end{document}