\documentclass[tp]{lcc}

\usepackage{hyperref}

\codigo{}
\materia{Fundamentos de Robótica Móvil}
\titulo{Transformaciones}

\soluciones
\commentstrue

\newcommand{\memcached}[0]{\texttt{memcached}}

\usepackage{biblatex}
%\addbibresource{refs.bib}

\begin{document}
\maketitle

\section{Entrega}
Se deberá entregar un informe realizado en Lyx o latex explicando los pasos realizados y analizando los resultados obtenidos.

\section{Introducción}
Esta práctica es para ponerse a punto con transformaciones.

\section{Rotaciones 3D}
Para el sistema de coordenadas canónico de un robot móvil (x: hacia adelante, y: hacia la izquierda, z: hacia arriba) dibujar el sistema de coordenadas resultante luego de aplicar la rotación dada.

\begin{problema}
    
    
\end{problema}




\end{document}
