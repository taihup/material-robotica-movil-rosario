\documentclass{article}
\usepackage{amsmath,amssymb,graphicx,hyperref}
\usepackage[margin=1in]{geometry}

\title{CSE 571 - Robotics \\ Homework 2 - EKF and RRT}
\author{Due Thursday May 16th @ 11:59pm}
\date{}

\begin{document}

\maketitle

\section*{Collaboration Policy}
Students can discuss questions, but each student MUST write up their own solution, and code their own solution. We will be checking code/PDFs for plagiarism.

\section*{Late Policy}
This assignment may be handed in up to 5 days late (Tuesday May 19th @ 11:59pm), at a penalty of 10\% of the maximum grade per day.

\section{Extended Kalman Filter}

\subsection{Jacobian Derivation}

\textbf{Hint:}
\begin{enumerate}
    \item Take a look at the relevant sections (7.4, 10.2) in the book \cite{thrun2005probabilistic}.
    \item Make sure you get the derivations correct before proceeding to implementation.
\end{enumerate}

\begin{figure}[h]
    \centering
    \includegraphics[width=0.5\textwidth]{odometry_model.png}
    \caption{Odometry-based motion model}
    \label{fig:odometry}
\end{figure}

\subsubsection{Motion Model Jacobian [5 points]}
We will reuse the odometry motion model from HW1 (see Figure \ref{fig:odometry} and Sec. 5.4 in the book \cite{thrun2005probabilistic}). The state of the robot is its 2D position and orientation: $s_{t}=[x_{t},y_{t},\theta_{t}]$. The control to the robot is $u_{t}=[\delta_{rot1},\delta_{trans},\delta_{rot2}]$, i.e. the robot rotates by $\delta_{rot1}$, drives straight forward $\delta_{trans}$, then rotates again by $\delta_{rot2}$.

The equations for the motion model $s_{t}=g(u_{t},s_{t-1})$ are as follows:

\begin{align*}
    x_{t} &= x_{t-1}+\delta_{trans}*\cos(\theta_{t-1}+\delta_{rot1}) \\
    y_{t} &= y_{t-1}+\delta_{trans}*\sin(\theta_{t-1}+\delta_{rot1}) \\
    \theta_{t} &= \theta_{t-1}+\delta_{rot1}+\delta_{rot2}
\end{align*}

Your task is to derive the Jacobian matrix $G$, a matrix consisting of the first-order derivative of $s_{t}$ with respect to the previous state $s_{t-1}$, as well as $V$, a matrix consisting of the first-order derivative of $s_{t}$ with respect to the control input $u_{t}$. Since $s_{t},s_{t-1}$ and $u_{t}$ are all 3-dimensional, $G$ and $V$ are both $3\times 3$ matrices.

In short, $s_{t+1}=[x_{t+1},y_{t+1},\theta_{t+1}]$ is the prediction of the motion model. Derive the Jacobians of $g$ with respect to the state $G=\dfrac{\partial g}{\partial s}$ and control $V=\dfrac{\partial g}{\partial u}$:

\[
G=\begin{pmatrix}
\dfrac{\partial x^{\prime}}{\partial x} & \dfrac{\partial x^{\prime}}{\partial y} & \dfrac{\partial x^{\prime}}{\partial \theta} \\
\dfrac{\partial y^{\prime}}{\partial x} & \dfrac{\partial y^{\prime}}{\partial y} & \dfrac{\partial y^{\prime}}{\partial \theta} \\
\dfrac{\partial \theta^{\prime}}{\partial x} & \dfrac{\partial \theta^{\prime}}{\partial y} & \dfrac{\partial \theta^{\prime}}{\partial \theta} \\
\end{pmatrix}
\quad
V=\begin{pmatrix}
\dfrac{\partial x^{\prime}}{\partial \delta_{rot1}} & \dfrac{\partial x^{\prime}}{\partial \delta_{trans}} & \dfrac{\partial x^{\prime}}{\partial \delta_{rot2}} \\
\dfrac{\partial y^{\prime}}{\partial \delta_{rot1}} & \dfrac{\partial y^{\prime}}{\partial \delta_{trans}} & \dfrac{\partial y^{\prime}}{\partial \delta_{rot2}} \\
\dfrac{\partial \theta^{\prime}}{\partial \delta_{rot1}} & \dfrac{\partial \theta^{\prime}}{\partial \delta_{trans}} & \dfrac{\partial \theta^{\prime}}{\partial \delta_{rot2}} \\
\end{pmatrix}
\]

\subsubsection{Observation Model Jacobian [5 points]}
Assume there is a landmark $m$ at location $(x_{m},y_{m})$. The robot receives two measurements of the bearing angle $\phi$ and the landmark, the range $r$, where

\begin{align*}
    \phi &= \texttt{atan2}(y_{m}-y_{t},x_{m}-x_{t})-\theta_{t} \\
    r &= \sqrt{(x_{m}-x_{t})^{2}+(y_{m}-y_{t})^{2}}
\end{align*}

Derive Jacobian matrix $H_{s}$, derivative of the measurements with respect to the robot state $s_{t}=[x_{t},y_{t},\theta_{t}]$ and the Jacobian matrix $H_{m}$, derivative of the measurements with respect to the landmark position $(x_{m},y_{m})$.

\subsection{EKF Localization and Mapping [40 points]}
In the programming component of this assignment, you will first implement an Extended Kalman Filter (EKF) (Fig. \ref{fig:ekf}) to localize a robot based on landmarks. Then, you will add landmarks to the state and build a map while localizing the robot, thus performing simultaneous localization and mapping (SLAM).

\begin{figure}[h]
    \centering
    \includegraphics[width=0.6\textwidth]{ekf.png}
    \caption{Extended Kalman Filter}
    \label{fig:ekf}
\end{figure}

\subsection*{Code Overview}
The starter code for EKF is under the \texttt{ekf\_slam} folder. The conda environment you installed for HW1 should suffice for this homework.

\subsection*{Command-Line Interface}
First, run \texttt{cd ekf\_slam} to enter the sub-folder. To visualize the robot in the soccer field environment, run:

\begin{verbatim}
$ python localization.py --plot none
\end{verbatim}

The blue line traces out the robot's position, which is a result of noisy actions. The green line traces the robot's position assuming that actions weren't noisy. After you implement a filter, the filter's estimate of the robot's position will be drawn in red.

\begin{verbatim}
$ python localization.py --plot ekf_slam
\end{verbatim}

Typical commands would be like:

\begin{verbatim}
# problem a
$ python localization.py ekf_slam --plot
# problem b
$ python localization.py ekf_slam --multi_run 10 --data-factor 0.1 --filter-factor 0.1
# problem c
$ python localization.py ekf_slam --multi_run 10 --motion-factor 0.1 --observation-factor 1.0
\end{verbatim}

To see other command-line flags available to you, run:

\begin{verbatim}
$ python localization.py -h
\end{verbatim}

\textbf{Hints:}
\begin{itemize}
    \item Make sure to call \texttt{utils.minimized\_angle} any time an angle or angle difference could exceed $[-\pi,\pi]$.
    \item Turn off plotting for a significant speedup.
\end{itemize}

\subsection*{Task}
\begin{enumerate}
    \item Fill in the Jacobian matrices G, V, R and H in \texttt{ekf\_slam.py} using your derivation in 1.1.
    \item Implement the EKF update (Fig. \ref{fig:ekf}) in \texttt{ExtendedKalmanFilterSLAM.update} in \texttt{ekf\_slam.py}. Make sure to keep track of the set of landmark IDs that the robot has observed.
\end{enumerate}

Answer the following questions in your writeup:
\begin{enumerate}[label=(\alph*)]
    \item Under the default noise parameters, $\alpha=\beta=1$, plot the robot's path and the landmark locations estimated by EKF and compare it with the ground truth path and landmarks.
    \item Plot the position error of the robot state and the landmark positions as the data factor (\texttt{--data-factor}) and the filter noise factor (\texttt{--filter-factor}) vary \textbf{simultaneously} over $[0.01,0.03,0.1,0.3,1.0]$. Keep \texttt{data-factor == filter-factor}, so there are 5 data points. For each data point, run 10 trials with different random seeds (\texttt{--multi\_run 10}) and plot the mean and standard deviation. Does EKF estimate the map accurately? Discuss anything interesting you observe.
    \item Plot the position error of robot state and the landmark positions as the motion factor (\texttt{--motion-factor}) and observation factor (\texttt{--observation-factor}) vary \textbf{independently} over $[0.1,0.3,1.0]$. There are 9 data points in total. Run 10 trials per value with different random seeds (\texttt{--multi\_run 10}) and plot the mean and standard deviation. Discuss anything interesting you observe.
\end{enumerate}

\section{Rapidly-exploring Random Tree}
In this part, you are provided with a robot arm that has 2 links is able to move in a 2D plane only. Your task is to implement Rapidly-exploring Random Tree (RRT), a classical sampling-based motion planning algorithms named after its data structure, and study the parameters that govern its behaviors.

\subsection{Code Overview}
The starter code for RRT is under the \texttt{rrt} folder. Here is a brief overview.

\subsection{RRT Implementation [50 points]}
Implement a RRT planner for the 2D robot in \texttt{RRTPlanner.py} by filling in \texttt{Plan} and \texttt{extend} functions. Your results from a successful RRT implementation should be comparable to the following results:

\begin{verbatim}
$ python plan.py -o 0 --seed 0
...
cost: 198.84383834015168

$ python plan.py -o 2 --seed 0
...
cost: 170.48992200228264
\end{verbatim}

If you turn on the \texttt{-v} flag, i.e. \texttt{python plan.py --seed 0 -v}, you should see a plot similar to Fig. \ref{fig:rrt}.

\begin{figure}[h]
    \centering
    \includegraphics[width=0.6\textwidth]{rrt_example.png}
    \caption{Example RRT path in configuration space.}
    \label{fig:rrt}
\end{figure}

\textbf{Answer the following questions.}

Note that since RRT is non-deterministic, you will need to provide statistical results, i.e. mean and standard deviation over \textbf{at least 5 runs} with different random seeds specified by \texttt{--seed}.

The \texttt{-o} flag specifies the number of obstacles in the environment. Please report your results with \texttt{-o 2}. You can use other values for debugging.

\begin{enumerate}
    \item Bias the sampling to pick the goal with $5\%$, $20\%$ probability. Report the performance (cost and time). For each setting, include at least one figure like Fig. \ref{fig:rrt} showing the RRT tree in configuration space. Which value would you pick in practice?
    \item You can view the robot as a point that can move in arbitrarily directions in the configuration space. In other words, the states can be interpolated via a straight line (see Fig. \ref{fig:extend} for an illustration). Compare two strategies for the \texttt{extend()} function:
    \begin{itemize}
        \item the nearest neighbor extends only half way to the sampled point (i.e. step size $\eta=0.5$)
        \item the nearest neighbor extends all the way till the sampled point (i.e. step size $\eta=1$)
    \end{itemize}
    Report the performance (cost, time) for each setting. Include at least one figure showing the final state of the RRT tree. Which strategy would you employ in practice?
    \item Discuss any challenges you faced and describe your debugging process.
\end{enumerate}

\begin{figure}[h]
    \centering
    \includegraphics[width=0.5\textwidth]{extend.png}
    \caption{Visualization of \texttt{extend()} with step size $\eta$, which controls the ratio of the distance from $x_{\text{new}}$ (the state to be added) to $x_{\text{near}}$ and the distance from $x_{\text{rand}}$ (the sampled state) to $x_{\text{near}}$. If $\eta=1$, $x_{\text{new}}=x_{\text{rand}}$.}
    \label{fig:extend}
\end{figure}

\textbf{Hint:}
Check out these useful functions that you should use to simplify your implementation:
\begin{itemize}
    \item In \texttt{RRTTree.py}:
    \begin{itemize}
        \item \texttt{AddVertex}
        \item \texttt{AddEdge}
        \item \texttt{GetNearestVertex}
    \end{itemize}
    \item In \texttt{ArmEnvironment.py}:
    \begin{itemize}
        \item \texttt{compute\_distance}
        \item \texttt{goal\_criterion}
        \item \texttt{edge\_validity\_checker}
    \end{itemize}
\end{itemize}

\section{Submission}
We will be using the Canvas for submission of the assignments. Please submit the written assignment answers as a PDF. For the code, submit a zip file of the entire working directory.

\begin{thebibliography}{9}
\bibitem{thrun2005probabilistic}
Thrun, Burgard and Fox (2005), \textit{Probabilistic Robotics}.
\end{thebibliography}

\end{document}