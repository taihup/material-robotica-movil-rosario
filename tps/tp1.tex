\documentclass[tp]{lcc}

\usepackage{hyperref}

\codigo{R-322}
\materia{Fundamentos de Robótica Móvil}
\titulo{Triangulación estéreo}

\soluciones
\commentstrue

\newcommand{\memcached}[0]{\texttt{memcached}}

\usepackage{biblatex}
%\addbibresource{refs.bib}

\begin{document}
\maketitle

\section{Introducción}

EL objetivo del tp es que se realicen los pasos básicos para poder triangular y proyectar puntos con una cámara estéreo. 
Para esto se trabajará con datos reales.

\section{Datos}
Para trabajar se utilizarán los siguientes datos:

\section{Calibración}
Antes de comenzar a trabajar con un dataset se debe realizar una calibración de los sensores. En este trabajo solo deben calibrar una cámara estéreo.

Para la calibración se puede utilizar cualquiera de las aplicaciones que se detallan a continuación:
\begin{itemize}
	\item OpenCV tutorial\_camera\_calibration: \href{https://docs.opencv.org/4.x/d4/d94/tutorial_camera_calibration.html}{https://docs.opencv.org/4.x/d4/d94/tutorial\_camera\_calibration.html}
	\item ROS camera\_calibration: \href{http://wiki.ros.org/camera_calibration/Tutorials/StereoCalibration}{http://wiki.ros.org/camera\_calibration/Tutorials/StereoCalibration}
	\item Camera Calibration Toolbox for Matlab: \href{https://www.cs.toronto.edu/pub/psala/VM/cameraCalibrationExample.html}{https://www.cs.toronto.edu/pub/psala/VM/cameraCalibrationExample.html}
\end{itemize}



Desarrollar un programa que lea un par de imágenes estéreo cualquiera y realice los siguientes pasos:

\section{Rectificación de imágenes}
Para rectificar las imágenes deberá hacer uso de la librería OpenCV. EN particular deberá utilizar las funciones: cv::stereoRectify(), cv::initUndistortRectifyMap() y remap().


\section{Extracción de feaures: Keypoints y descriptores}

\begin{enumerate}
	\item Detectar Keypoitns: Fast, ORB, GFTT, etc. 
	\item Extraer descriptores: BRIEF, ORB, etc.
\end{enumerate}

\section{Búsqueda de correspondencias visuales}
Realizar la búsqueda de correspondencias entre los feature de ambas imágenes (\emph{matching}). Para esto deberá utilizar la función cv::BFMatcher::BFMatcher(). 

\section{Triangulación de puntos 3D}
Dados los matches encontrados, realice la triangulación de los puntos utilizando la función: 

\section{Filtrado de correspondencias espúreas}
Aplicar RANSAC para filtrar los matches espúreos y calcular la matríz fundamental. Para esto puede utilizar la función cv::findHomography().

\section{Calcular el mapa de disparidad}
Calcular el mapa de disparidad para esto  deberá utilizar la función cv::StereoMatcher::compute() de OpenCV. También utilizar la librería Elas.

\section{Reconstrucción 3D}
Utilizando el mapa de disparidad realice una reconstrucción densa de la escena observada. Para esto puede utilizar la matríz de retro-proyección Q.

\section{Calcular la matriz Esencial}
Utilizando cv::recoverPose() estime la transformación entre ambas cámaras. Para esto deberá calcular la matríz esencial utilizando la función cv::findEssentialMat().


\printbibliography

\end{document}
