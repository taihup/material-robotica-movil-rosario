\documentclass[tp]{lcc}

\usepackage{hyperref}

\codigo{}
\materia{Fundamentos de Robótica Móvil}
\titulo{Transformaciones}

\soluciones
\commentstrue

\newcommand{\memcached}[0]{\texttt{memcached}}

\usepackage{biblatex}
%\addbibresource{refs.bib}

\begin{document}
\maketitle

\section{Entrega}
Se deberá proveer un repositorio git y junto con las instrucciones de compilación en un archivo README.md.
Además, se deberá entregar un informe realizado en Lyx o latex explicando los pasos realizados y analizando los resultados obtenidos.

\section{Introducción}

El objetivo del tp es que se realicen los pasos básicos para poder triangular y proyectar puntos con una cámara estéreo. 
Para esto se trabajará con datos reales.


\end{document}
