\documentclass[tp]{lcc}

% add latex preamble
% para la bibliografía se requiere biber y configurar texstudio

% Latex packages
\usepackage[utf8]{inputenc}
\usepackage[T1]{fontenc} % para copiar acentos en español del pdf y permite acentos en las notas
\usepackage[spanish]{babel}
\usepackage[per-mode = symbol]{siunitx} % para manejar las unidades
\usepackage{multimedia} % to add videos with \movie command
\usepackage{multirow}
\usepackage{graphicx}
\usepackage{xcolor}
\usepackage{amsmath} % bmatrix
\usepackage[makeroom]{cancel} % \cancel to cancel terms in math equations
\renewcommand{\CancelColor}{\color{red}} % set red color for \cancel command
\usepackage[caption=false]{subfig} % caption = false elimina la palabra "Figura" del caption
\usepackage{import} % para el comando import (se usa para pdf_tex)
\captionsetup[subfigure]{labelformat=empty} % remover el indice del caption de la subfigura
\usepackage{booktabs} % \toprule \midrule \bottomrule
\usepackage[backend=biber]{biblatex} % set biber to format references. Must configure Biber in Texstudio
\usepackage{csquotes} % to remove warning triggered by biblatex and babel
\usepackage{algorithm} % to put captions to the algorithmics environmets
\usepackage{algpseudocode} % to write algorithm
\usepackage{tikz} % to use tikz
\usetikzlibrary{fit} % to fit a node around other nodes in tikz
\usepackage[export]{adjustbox} % valign in subfloat
\usepackage{colortbl} % to paint cells in a table

% Color commands for annotations
\newcommand\TODO[1]{\textbf{\textcolor{red}{#1}}} %  TODO notes

% Graphic paths
\graphicspath{{./images/}}

% listings configuration for C code
\usepackage{listings} % code
\definecolor{commentgreen}{RGB}{2,112,10}
\definecolor{eminence}{RGB}{108,48,130}
\definecolor{weborange}{RGB}{255,165,0}
\definecolor{frenchplum}{RGB}{129,20,83}

\lstset{ % spanish characters for listings package
	inputencoding=latin1,
    columns=fullflexible,
	breaklines=true,
	tabsize=2,
	showstringspaces=false,
	basicstyle=\ttfamily,
	backgroundcolor=\color{lightgray}, % Choose background color
	literate={á}{{\'a}}1
	{ã}{{\~a}}1
	{é}{{\'e}}1
	{ó}{{\'o}}1
	{í}{{\'i}}1
	{ñ}{{\~n}}1
	{¡}{{!`}}1
	{¿}{{?`}}1
	{ú}{{\'u}}1
	{Í}{{\'I}}1
	{Ó}{{\'O}}1
    {-}{-}1
}

\lstdefinestyle{cpp}{ % spanish characters for listings package
    language=C++,
   	commentstyle=\color{commentgreen},
    keywordstyle=\color{eminence},
    stringstyle=\color{red},
    emph={int,char,double,float,unsigned,void,bool},
    emphstyle={\color{blue}}
}

\lstdefinestyle{bash}{ % spanish characters for listings package
	language=Bash
}

\lstdefinestyle{xml}{
	language=XML,
	morekeywords={encoding,xs:schema,xs:element,xs:complexType,xs:sequence,xs:attribute}
}

\lstdefinestyle{cmake}{
	language=make, % there is no cmake support in listings
}

\lstdefinestyle{python}{
    language=python,
}


%%%%% PARA QUE EN LAS TABLAS SE PUEDA PONER UN SALTO DE LINEA DENTRO DE UNA CELDA
\newcommand{\specialcell}[2][c]{%
    \begin{tiny}
        \begin{tabular}[#1]{@{}c@{}}#2\end{tabular}  
    \end{tiny}
}
%%%%%%%%%%%%%%%%%%%%%%%%%%%%%%%%%%%%%%%%%%%%%%%%%%%%%%%%%%%%%%%%%%%%%%%%

%%%%% PARA QUE LAS TABLAS TENGAN TODAS LAS COLUMNAS CENTRADAS Y DE IGUAL TAMAÑO
\usepackage{tabularx}
\renewcommand{\tabularxcolumn}[1]{>{\centering\arraybackslash}m{#1}}
%%%%%%%%%%%%%%%%%%%%%%%%%%%%%%%%%%%%%%%%%%%%%%%%%%%%%%%%%%%%%%%%%%%%%%%%



% add math preamble
\usepackage{amsmath}
\usepackage{amssymb}
\usepackage{amsopn}
\usepackage{mathtools}

% math
\renewcommand{\vec}[1]{\boldsymbol{\mathbf{#1}}}
\newcommand{\norm}[1]{\lVert#1\rVert}

% Declare arg max and arg min functionss
\DeclareMathOperator*{\argmax}{arg\,max}
\DeclareMathOperator*{\argmin}{arg\,min}

% Homogeneous decoration function
\newcommand{\homo}[1]{\dot{#1}}


% Declare projection as math function
\DeclareMathOperator{\proj}{proj}
\newcommand{\fromCoord}[2]{{#1}_\mathrm{#2}}
\newcommand{\toCoord}[2]{\prescript{\mathrm{#2}}{}{#1}}
\newcommand{\worldCoordSystem}{\mathrm{w}}
\newcommand{\bodyCoordSystem}{\mathrm{B}}
\newcommand{\cameraCoordSystem}{\mathrm{c}}
\newcommand{\point}{\vec{p}}
\newcommand{\worldPoint}{\toCoord{\point}{\worldCoordSystem}}
\newcommand{\imagePoint}{\vec{u}}
\newcommand{\cameraPoint}{\toCoord{\point}{\cameraCoordSystem}}
\newcommand{\homoWorldPoint}{\toCoord{\homo{\point}}{\worldCoordSystem}}
\newcommand{\homoImagePoint}{\homo{\imagePoint}}
\newcommand{\homoCameraPoint}{\toCoord{\homo{\point}}{\cameraCoordSystem}}
\newcommand{\measurement}{\vec{z}}
\newcommand{\prediction}{\hat{\vec{z}}}
\newcommand{\seMatrix}{\vec{\xi}}
\newcommand{\transform}[2]{\toCoord{\fromCoord{\seMatrix}{#2}}{#1}}
\newcommand{\pointCoord}[1]{\toCoord{\point}{#1}}
\newcommand{\rotation}{\vec{R}}
\newcommand{\rotationCoord}[2]{\toCoord{\fromCoord{\rotation}{#2}}{#1}}
\newcommand{\translation}{\vec{t}}
\newcommand{\translationCoord}[2]{\toCoord{\fromCoord{\translation}{#2}}{#1}}
\newcommand{\intrinsicMatrix}{\vec{K}}
\newcommand{\principalPoint}{\vec{c}}
\newcommand{\reprojectionError}{u}
\newcommand{\projectionMatrix}{\vec{P}}
\newcommand{\cameraCenter}{\vec{o}}
\newcommand{\essentialMatrix}{\vec{E}}
\newcommand{\inverse}[1]{{#1}^{-1}}

% Motion model
\newcommand{\position}{\vec{p}}
\newcommand{\orientationQuaternion}{\vec{q}}
\newcommand{\predictedPosition}{\hat{\vec{p}}}
\newcommand{\predictedOrientationQuaternion}{\hat{\vec{q}}}
\newcommand{\linearVelocity}{\vec{v}}
\newcommand{\angularVelocity}{\vec{\omega}}

\DeclareMathOperator{\slerpOp}{slerp}
\newcommand{\slerp}[1]{\slerpOp{\left( #1 \right)}}

% Map structure
\newcommand{\map}{M}
\newcommand{\keyframesSet}{K}
\newcommand{\mapPointsSet}{P}
\newcommand{\observedMapPoints}{O}
\newcommand{\covisibilityKeyframes}{CK}
\newcommand{\localMap}{local\_map}



% Bundle Adjutment
\newcommand{\update}{\vec{\delta}}
\newcommand{\incremental}{\hat{\update}}


% Loop Closure names

% scaled operators and letters to fancy view
\newcommand{\sminus}{\scalebox{0.5}[1.0]{$-$}}
\newcommand{\splus}{\scalebox{0.6}[0.6]{$+$}}
\newcommand{\curr}{c}
\newcommand{\sind}[1]{\scalebox{0.6}[0.6]{$#1$}}
\newcommand{\ind}[1]{\scalebox{0.7}[0.7]{$#1$}}

\newcommand{\keyframe}{\vec{K}}
\newcommand{\bowVector}{\vec{v}}
\newcommand{\lcError}{\vec{\Omega}}
\newcommand{\relativeTransformation}{\seMatrix}
\DeclareMathOperator{\interpolate}{interpolate}

\newcommand{\relativeMotion}{\vec{\delta}}
\newcommand{\groundTruth}[1]{{#1}^{*}}



% definición del operador rot()
\DeclareMathOperator{\rotationOp}{rot}
\newcommand{\getRotation}[1]{\rotationOp{\left( #1 \right)}}

\DeclareMathOperator{\translationOp}{trans}
\newcommand{\getTranslation}[1]{\translationOp{\left( #1 \right)}}









\codigo{R-521}
\materia{Fundamentos de Robótica Móvil}
\titulo{Transformaciones}

\soluciones
\commentstrue


\usepackage{biblatex}
%\addbibresource{refs.bib}

\begin{document}
\maketitle

\section{Entrega}
Se deberá proveer un repositorio git y junto con las instrucciones de compilación en un archivo README.md.
Además, se deberá entregar un informe realizado en Lyx o latex explicando los pasos realizados y analizando los resultados obtenidos.

\section{Introducción}

El objetivo del tp es que se realicen los pasos básicos para poder triangular y proyectar puntos con una cámara estéreo. 
Para esto se trabajará con datos reales.

\section{Datos}
Para trabajar se utilizarán los siguientes datos:

\section{Calibración}
Antes de comenzar a trabajar con un dataset se debe realizar una calibración de los sensores. En este trabajo solo deben calibrar una cámara estéreo.

Para la calibración se puede utilizar cualquiera de las aplicaciones que se detallan a continuación:
\begin{itemize}
	\item OpenCV tutorial\_camera\_calibration: \href{https://docs.opencv.org/4.x/d4/d94/tutorial_camera_calibration.html}{https://docs.opencv.org/4.x/d4/d94/tutorial\_camera\_calibration.html}
	\item ROS camera\_calibration: \href{http://wiki.ros.org/camera_calibration/Tutorials/StereoCalibration}{http://wiki.ros.org/camera\_calibration/Tutorials/StereoCalibration}
	\item Kalibr: \href{https://github.com/ethz-asl/kalibr}{https://github.com/ethz-asl/kalibr}
	\item Camera Calibration Toolbox for Matlab: \href{https://www.cs.toronto.edu/pub/psala/VM/cameraCalibrationExample.html}{https://www.cs.toronto.edu/pub/psala/VM/cameraCalibrationExample.html}
\end{itemize}

Desarrollar un programa que lea un par de imágenes estéreo cualquiera y realice los siguientes pasos:

\section{Rectificación de imágenes}
Con los parámetros íntrinsecos y extrínsecos, rectificar un par imágenes haciendo uso de la librería OpenCV. Para esto se deberan utilizar las funciones: cv::stereoRectify(), cv::initUndistortRectifyMap() y remap().

\section{Extracción de feaures: Keypoints y descriptores}

\begin{enumerate}
	\item Detectar Keypoitns: FAST, ORB, SIFT, SURF, GFTT, BRISK, etc.
	\item Extraer descriptores: BRIEF, ORB, BRISK, etc.
\end{enumerate}

\section{Búsqueda de correspondencias visuales}
Realizar la búsqueda de correspondencias entre los feature de ambas imágenes (\emph{matching}). Para esto deberá utilizar la función cv::BFMatcher::BFMatcher(). 

\section{Triangulación de puntos 3D}
Dados los matches encontrados, realice la triangulación de los puntos utilizando la función: cv::sfm::triangulatePoints().

\section{Filtrado de correspondencias espúreas}
Aplicar RANSAC para filtrar los matches espúreos y calcular la matríz fundamental. Para esto puede utilizar la función cv::findHomography().

\section{Calcular el mapa de disparidad}
Calcular el mapa de disparidad para esto  deberá utilizar la función cv::StereoMatcher::compute() de OpenCV. También utilizar la librería LIBELAS (\href{http://www.cvlibs.net/software/libelas/}{http://www.cvlibs.net/software/libelas/})

\section{Reconstrucción 3D densa}
Utilizando el mapa de disparidad realice una reconstrucción densa de la escena observada utilizando la función cv::reprojectImageTo3D(). Para esto debe utilizar la matríz de reproyección Q retornada por la función cv::stereoRectify().

\section{Estimación de pose Monocular}
Utilizando cv::recoverPose() estimar la transformación entre la imagen izquierda y la imagen derecha. Para esto deberá calcular la matríz esencial utilizando la función cv::findEssentialMat().

\printbibliography

\end{document}
