\documentclass[tp]{lcc}

\usepackage{hyperref}

\codigo{R-322}
\materia{Fundamentos de Robótica Móvil}
\titulo{Triangulación estéreo}

\soluciones
\commentstrue

\newcommand{\memcached}[0]{\texttt{memcached}}

\usepackage{biblatex}
%\addbibresource{refs.bib}

\begin{document}
\maketitle

\section{Entrega}
Se deberá proveer un repositorio git y junto con las instrucciones de compilación en un archivo README.md.
Además, se deberá entregar un informe realizado en Lyx o latex explicando los pasos realizados y analizando los resultados obtenidos.

\section{Introducción}

El objetivo del tp es que se realicen los pasos básicos para poder triangular y proyectar puntos con una cámara estéreo. 
Para esto se trabajará con datos reales.

\section{Datos}
Para trabajar se utilizarán los siguientes datos:

\section{Calibración}
Antes de comenzar a trabajar con un dataset se debe realizar una calibración de los sensores. En este trabajo solo deben calibrar una cámara estéreo.

Para la calibración se puede utilizar cualquiera de las aplicaciones que se detallan a continuación:
\begin{itemize}
	\item OpenCV tutorial\_camera\_calibration: \href{https://docs.opencv.org/4.x/d4/d94/tutorial_camera_calibration.html}{https://docs.opencv.org/4.x/d4/d94/tutorial\_camera\_calibration.html}
	\item ROS camera\_calibration: \href{http://wiki.ros.org/camera_calibration/Tutorials/StereoCalibration}{http://wiki.ros.org/camera\_calibration/Tutorials/StereoCalibration}
	\item Kalibr: \href{https://github.com/ethz-asl/kalibr}{https://github.com/ethz-asl/kalibr}
	\item Camera Calibration Toolbox for Matlab: \href{https://www.cs.toronto.edu/pub/psala/VM/cameraCalibrationExample.html}{https://www.cs.toronto.edu/pub/psala/VM/cameraCalibrationExample.html}
\end{itemize}

Desarrollar un programa que lea un par de imágenes estéreo cualquiera y realice los siguientes pasos:

\section{Rectificación de imágenes}
Con los parámetros íntrinsecos y extrínsecos, rectificar un par imágenes haciendo uso de la librería OpenCV. Para esto se deberan utilizar las funciones: cv::stereoRectify(), cv::initUndistortRectifyMap() y remap().

\section{Extracción de feaures: Keypoints y descriptores}

\begin{enumerate}
	\item Detectar Keypoitns: FAST, ORB, SIFT, SURF, GFTT, BRISK, etc.
	\item Extraer descriptores: BRIEF, ORB, BRISK, etc.
\end{enumerate}

\section{Búsqueda de correspondencias visuales}
Realizar la búsqueda de correspondencias entre los feature de ambas imágenes (\emph{matching}). Para esto deberá utilizar la función cv::BFMatcher::BFMatcher(). 

\section{Triangulación de puntos 3D}
Dados los matches encontrados, realice la triangulación de los puntos utilizando la función: cv::sfm::triangulatePoints().

\section{Filtrado de correspondencias espúreas}
Aplicar RANSAC para filtrar los matches espúreos y calcular la matríz fundamental. Para esto puede utilizar la función cv::findHomography().

\section{Calcular el mapa de disparidad}
Calcular el mapa de disparidad para esto  deberá utilizar la función cv::StereoMatcher::compute() de OpenCV. También utilizar la librería LIBELAS (\href{http://www.cvlibs.net/software/libelas/}{http://www.cvlibs.net/software/libelas/})

\section{Reconstrucción 3D densa}
Utilizando el mapa de disparidad realice una reconstrucción densa de la escena observada utilizando la función cv::reprojectImageTo3D(). Para esto debe utilizar la matríz de reproyección Q retornada por la función cv::stereoRectify().

\section{Estimación de pose Monocular}
Utilizando cv::recoverPose() estimar la transformación entre la imagen izquierda y la imagen derecha. Para esto deberá calcular la matríz esencial utilizando la función cv::findEssentialMat().

\printbibliography

\end{document}
